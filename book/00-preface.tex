\chapter{Preface}
\label{preface}

This book is an
introduction to the practical tools of exploratory data analysis.
The organization of the book follows the process I use
when I start working with a dataset:

\begin{itemize}

\item Importing and cleaning: Whatever format the data is in, it
  usually takes some time and effort to read the data, clean and
  transform it, and check that everything made it through the
  translation process intact.
\index{cleaning}

\item Single variable explorations: I usually start by examining one
  variable at a time, finding out what the variables mean, looking
  at distributions of the values, and choosing appropriate
  summary statistics.
\index{distribution}

\item Pair-wise explorations: To identify possible relationships
  between variables, I look at tables and scatter plots, and compute
  correlations and linear fits.
\index{correlation}
\index{linear fit}

\item Multivariate analysis: If there are apparent relationships
  between variables, I use multiple regression to add control variables
  and investigate more complex relationships.
\index{multiple regression}
\index{control variable}

\item Estimation and hypothesis testing: When reporting statistical
  results, it is important to answer three questions: How big is
  the effect?  How much variability should we expect if we run the same
  measurement again?  Is it possible that the apparent effect is
  due to chance?
\index{estimation}
\index{hypothesis testing}

\item Visualization: During exploration, visualization is an important 
  tool for finding possible relationships and effects.  Then if an
  apparent effect holds up to scrutiny, visualization is an effective
  way to communicate results.
\index{visualization}

\end{itemize}

This book takes a computational approach, which has several
advantages over mathematical approaches:
\index{computational methods}

\begin{itemize}

\item I present most ideas using Python code, rather than
  mathematical notation.  In general, Python code is more readable;
  also, because it is executable, readers can download it, run it,
  and modify it.

\item Each chapter includes exercises readers can do to develop
  and solidify their learning.  When you write programs, you
  express your understanding in code; while you are debugging the
  program, you are also correcting your understanding.
\index{debugging}

\item Some exercises involve experiments to test statistical
  behavior.  For example, you can explore the Central Limit Theorem
  (CLT) by generating random samples and computing their sums.  The
  resulting visualizations demonstrate why the CLT works and when
  it doesn't.
\index{Central Limit Theorem}
\index{CLT}

\item Some ideas that are hard to grasp mathematically are easy to
  understand by simulation.  For example, we approximate p-values by
  running random simulations, which reinforces the meaning of the
  p-value.
\index{p-value}

\item Because the book is based on a general-purpose programming
  language (Python), readers can import data from almost any source.
  They are not limited to datasets that have been cleaned and
  formatted for a particular statistics tool.

\end{itemize}

The book lends itself to a project-based approach.  In my class,
students work on a semester-long project that requires them to pose a
statistical question, find a dataset that can address it, and apply
each of the techniques they learn to their own data.

To demonstrate my approach to statistical analysis, the book
presents a case study that runs through all of the chapters.  It uses
data from two sources:

\begin{itemize}

\item The National Survey of Family Growth (NSFG), conducted by the
  U.S. Centers for Disease Control and Prevention (CDC) to gather
  ``information on family life, marriage and divorce, pregnancy,
  infertility, use of contraception, and men's and women's health.''
  (See \url{http://cdc.gov/nchs/nsfg.htm}.)

\item The Behavioral Risk Factor Surveillance System (BRFSS),
  conducted by the National Center for Chronic Disease Prevention and
  Health Promotion to ``track health conditions and risk behaviors in
  the United States.''  (See \url{http://cdc.gov/BRFSS/}.)

\end{itemize}

Other examples use data from the IRS, the U.S. Census, and
the Boston Marathon.

This second edition of {\it Think Stats} includes the chapters from
the first edition, many of them substantially revised, and new
chapters on regression, time series analysis, survival analysis,
and analytic methods.  The previous edition did not use pandas,
SciPy, or StatsModels, so all of that material is new.


\section{How I wrote this book}

When people write a new textbook, they usually start by
reading a stack of old textbooks.  As a result, most books
contain the same material in pretty much the same order.

I did not do that.  In fact, I used almost no printed material while I
was writing this book, for several reasons:

\begin{itemize}

\item My goal was to explore a new approach to this material, so I didn't
want much exposure to existing approaches.

\item Since I am making this book available under a free license, I wanted
to make sure that no part of it was encumbered by copyright restrictions.

\item Many readers of my books don't have access to libraries of
printed material, so I tried to make references to resources that are
freely available on the Internet.

\item Some proponents of old media think that the exclusive
use of electronic resources is lazy and unreliable.  They might be right
about the first part, but I think they are wrong about the second, so
I wanted to test my theory.

% http://www.ala.org/ala/mgrps/rts/nmrt/news/footnotes/may2010/in_defense_of_wikipedia_bonnett.cfm

\end{itemize}

The resource I used more than any other is Wikipedia.  In general, the
articles I read on statistical topics were very good (although I made
a few small changes along the way).  I include references to Wikipedia
pages throughout the book and I encourage you to follow those links;
in many cases, the Wikipedia page picks up where my description leaves
off.  The vocabulary and notation in this book are generally
consistent with Wikipedia, unless I had a good reason to deviate.
Other resources I found useful were Wolfram MathWorld and 
the Reddit statistics forum, \url{http://www.reddit.com/r/statistics}.


\section{Using the code}
\label{code}

The code and data used in this book are available from
\url{https://github.com/AllenDowney/ThinkStats2}.  Git is a version
control system that allows you to keep track of the files that
make up a project.  A collection of files under Git's control is
called a {\bf repository}.  GitHub is a hosting service that provides
storage for Git repositories and a convenient web interface.
\index{repository}
\index{Git}
\index{GitHub}

The GitHub homepage for my repository provides several ways to
work with the code:

\begin{itemize}

\item You can create a copy of my repository
on GitHub by pressing the {\sf Fork} button.  If you don't already
have a GitHub account, you'll need to create one.  After forking, you'll
have your own repository on GitHub that you can use to keep track
of code you write while working on this book.  Then you can
clone the repo, which means that you make a copy of the files
on your computer.
\index{fork}

\item Or you could clone
my repository.  You don't need a GitHub account to do this, but you
won't be able to write your changes back to GitHub.
\index{clone}

\item If you don't want to use Git at all, you can download the files
in a Zip file using the button in the lower-right corner of the
GitHub page.

\end{itemize}

All of the code is written to work in both Python 2 and Python 3
with no translation.

I developed this book using Anaconda from
Continuum Analytics, which is a free Python distribution that includes
all the packages you'll need to run the code (and lots more).
I found Anaconda easy to install.  By default it does a user-level
installation, not system-level, so you don't need administrative
privileges.  And it supports both Python 2 and Python 3.  You can
download Anaconda from \url{http://continuum.io/downloads}.
\index{Anaconda}

If you don't want to use Anaconda, you will need the following
packages:

\begin{itemize}

\item pandas for representing and analyzing data,
  \url{http://pandas.pydata.org/};
\index{pandas}

\item NumPy for basic numerical computation, \url{http://www.numpy.org/};
\index{NumPy}

\item SciPy for scientific computation including statistics,
  \url{http://www.scipy.org/};
\index{SciPy}

\item StatsModels for regression and other statistical analysis,
\url{http://statsmodels.sourceforge.net/}; and
\index{StatsModels}

\item matplotlib for visualization, \url{http://matplotlib.org/}.
\index{matplotlib}

\end{itemize}

Although these are commonly used packages, they are not included with
all Python installations, and they can be hard to install in some
environments.  If you have trouble installing them, I strongly
recommend using Anaconda or one of the other Python distributions
that include these packages.
\index{installation}

After you clone the repository or unzip the zip file, you should have
a folder called {\tt ThinkStats2/code} with a file called {nsfg.py}.
If you run {nsfg.py}, it should read a data file, run some tests, and print a
message like, ``All tests passed.''  If you get import errors, it
probably means there are packages you need to install.

Most exercises use Python scripts, but some also use the IPython
notebook.  If you have not used IPython notebook before, I suggest
you start with the documentation at
\url{http://ipython.org/ipython-doc/stable/notebook/notebook.html}.
\index{IPython}

I wrote this book assuming that the reader is familiar with core Python,
including object-oriented features, but not pandas,
NumPy, and SciPy.  If you are already familiar with these modules, you
can skip a few sections.

I assume that the reader knows basic mathematics, including
logarithms, for example, and summations.  I refer to calculus concepts
in a few places, but you don't have to do any calculus.

If you have never studied statistics, I think this book is a good place
to start.  And if you have taken
a traditional statistics class, I hope this book will help repair the
damage.



---

Allen B. Downey is a Professor of Computer Science at 
the Franklin W. Olin College of Engineering in Needham, MA.




\section*{Contributor List}

If you have a suggestion or correction, please send email to 
{\tt downey@allendowney.com}.  If I make a change based on your
feedback, I will add you to the contributor list
(unless you ask to be omitted).
\index{contributors}

If you include at least part of the sentence the
error appears in, that makes it easy for me to search.  Page and
section numbers are fine, too, but not quite as easy to work with.
Thanks!

\small

\begin{itemize}

\item Lisa Downey and June Downey read an early draft and made many
corrections and suggestions.

\item Steven Zhang found several errors.

\item Andy Pethan and Molly Farison helped debug some of the solutions,
and Molly spotted several typos.

\item Andrew Heine found an error in my error function.

\item Dr. Nikolas Akerblom knows how big a Hyracotherium is.

\item Alex Morrow clarified one of the code examples.

\item Jonathan Street caught an error in the nick of time.

\item G\'{a}bor Lipt\'{a}k found a typo in the book and the relay race solution.

\item Many thanks to Kevin Smith and Tim Arnold for their work on
plasTeX, which I used to convert this book to DocBook.

\item George Caplan sent several suggestions for improving clarity.

\item Julian Ceipek found an error and a number of typos.

\item Stijn Debrouwere, Leo Marihart III, Jonathan Hammler, and Kent Johnson
found errors in the first print edition.

\item Dan Kearney found a typo.

\item Jeff Pickhardt found a broken link and a typo.

\item J\"{o}rg Beyer found typos in the book and made many corrections
in the docstrings of the accompanying code.

\item Tommie Gannert sent a patch file with a number of corrections.

\item Alexander Gryzlov suggested a clarification in an exercise.

\item Martin Veillette reported an error in one of the formulas for
Pearson's correlation.

\item Christoph Lendenmann submitted several errata.

\item Haitao Ma noticed a typo and and sent me a note.

\item Michael Kearney sent me many excellent suggestions.

\item Alex Birch made a number of helpful suggestions.

\item Lindsey Vanderlyn, Griffin Tschurwald, and Ben Small read an
  early version of this book and found many errors.

\item John Roth, Carol Willing, and Carol Novitsky performed technical
reviews of the book.  They found many errors and made many
helpful suggestions.

\item Rohit Deshpande found a typesetting error.

\item David Palmer sent many helpful suggestions and corrections.

\item Erik Kulyk found many typos.

\item Nir Soffer sent several excellent pull requests for both the
  book and the supporting code.

\item Joanne Pratt found a number that was off by a factor of 10.

% ENDCONTRIB

\end{itemize}

\normalsize

\clearemptydoublepage
