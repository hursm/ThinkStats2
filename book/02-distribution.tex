\chapter{분포 (Distribution)}
\label{descriptive}


\section{히스토그램}
\label{histograms}

변수를 기술하는 가장 좋은 방법중의 하나는 데이터셋에 나타나는 값과
각 값이 얼마나 나타나는지를 보고하는 것이다.
이러한 기술법을 변수 {\bf 분포(distribution)}라고 부른다.
\index{분포 (distribution)}

가장 일반적인 분포 표현은 {\bf 히스토그램(histogram)}으로 각 값의 
{\bf 빈도(frequency)}를 보여주는 그래프다. 이러한 맥락에서 ``빈도''는
값이 출현하는 횟수를 의미한다. 
\index{히스토그램 (histogram)} 
\index{빈도 (frequency)}
\index{딕셔러리 (dictionary)}

파이썬에서 빈도를 계산하는 효율적인 방법은 딕셔너리를 사용하는 것이다.
시퀀스 값 {\tt t}가 주어진 상태에서,
%
\begin{verbatim}
hist = {}
for x in t:
    hist[x] = hist.get(x, 0) + 1
\end{verbatim}

결과는 값을 빈도로 매칭하는 딕셔너리다. 대안으로 
{\tt collections} 모듈에 정의된 {\tt Counter} 클래스를 사용할 수 있다.


\begin{verbatim}
from collections import Counter
counter = Counter(t)
\end{verbatim}

결과는 {\tt Counter} 객체로 딕셔너리의 하위클래스가 된다.

또다른 선택지는 판다스 \verb"value_counts" 메쏘드를 사용하는 것으로 앞장에서 살펴봤다.
하지만, 이 책에서 히스토그램을 나타내는 Hist 클래스를 생성하고 Hist 클래스에서 동작하는 메쏘드를 제공한다.

\index{판다스 (pandas)}


\section{히스토그램 표현하기}
\index{히스토그램 (histogram)}
\index{Hist}

Hist 생성자는 시퀀스, 딕셔너리, 판다스 시리즈, 혹은 다른 Hist를 받을 수 있다.
다음과 같이 Hist 객체 인스턴스를 생성할 수 있다.

%
\begin{verbatim}
>>> import thinkstats2
>>> hist = thinkstats2.Hist([1, 2, 2, 3, 5])
>>> hist
Hist({1: 1, 2: 2, 3: 1, 5: 1})
\end{verbatim}

Hist 객체는 {\tt Freq} 메쏘드를 제공하는데 값을 받아 빈도를 반환한다.
\index{빈도 (frequency)}

%
\begin{verbatim}
>>> hist.Freq(2)
2
\end{verbatim}

꺾쇠 연산자도 동일한 것을 수행한다.
\index{꺾쇠 연산자 (bracket operator)}

%
\begin{verbatim}
>>> hist[2]
2
\end{verbatim}

만약 찾는 값이 없다면, 빈도는 0이다.

%
\begin{verbatim}
>>> hist.Freq(4)
0
\end{verbatim}

{\tt Values} 메쏘드는 Hist에 정렬되지 않는 리스트 값을 반환한다.
%
\begin{verbatim}
>>> hist.Values()
[1, 5, 3, 2]
\end{verbatim}

정렬된 값으로 루프를 돌리려면, 내장함수 {\tt sorted}를 사용할 수 있다.

%
\begin{verbatim}
for val in sorted(hist.Values()):
    print(val, hist.Freq(val))
\end{verbatim}

{\tt Items} 메쏘드를 사용해서 값-빈도(value-frequency) 짝을 반복처리할 수 있다.

%
\begin{verbatim}
for val, freq in hist.Items():
     print(val, freq)
\end{verbatim}


\section{히스토그램 그리기 (Plotting histograms)}
\index{pyplot}

\begin{figure}
% first.py
%\centerline{\includegraphics[height=2.5in]{figs/first_wgt_lb_hist.pdf}}
%\centerline{\includegraphics[height=2.50in]{figs/_test.eps}}
\caption{Histogram of the pound part of birth weight.}
\label{first_wgt_lb_hist}
\end{figure}

이 책에서 저자는 {\tt thinkplot.py} 모듈을 작성해서 Hists를 그리는 함수와 
{\tt thinkstats2.py}에 정의된 객체를 제공한다. {\tt pyplot}에 기반하고 있고 
{\tt matplotlib} 패키지의 일부다. 
{\tt matplotlib}을 설치하는 방법은 ~\ref{code}을 참조하세요.

\index{thinkplot}
\index{matplotlib}

{\tt thinkplot}으로 {\tt hist}를 그리기 위해서 다음을 시도해 보세요.

\index{Hist}

\begin{verbatim}
>>> import thinkplot
>>> thinkplot.Hist(hist)
>>> thinkplot.Show(xlabel='value', ylabel='frequency')
\end{verbatim}

\url{http://greenteapress.com/thinkstats2/thinkplot.html} 웹사이트에서
{\tt thinkplot}에 대한 문서를 참조할 수 있다.

\begin{figure}
% first.py
%\centerline{\includegraphics[height=2.5in]{figs/first_wgt_oz_hist.pdf}}
\caption{Histogram of the ounce part of birth weight.}
\label{first_wgt_oz_hist}
\end{figure}


\section{NSFG 변수}

이제 NSFG에 있는 데이터로 다시 돌아가자. 이장에 있는 코드는 {\tt first.py}다. 
코드를 다운로드하고 작업에 대한 정보는 ~\ref{code} 장을 참조하라.

새로운 데이터셋을 가지고 작업을 시작할 때, 한번에 하나씩 사용하려는 변수를 탐색하길 제안한다. 
시작하는 좋은 방법은 히스토그램을 그려보는 것이다.

\index{히스토그램 (histogram)}


~\ref{cleaning}에서 {\tt agepreg}를 백분년에서 년단위로 변환했고,
\verb"birthwgt_lb"와 \verb"birthwgt_oz"을 조합해서 \verb"totalwgt_lb" 한 단위량으로 만들었다.
이번 절에서 히스토그램의 몇가지 기능을 시연하기 위해서 이 변수들을 사용한다.


\begin{figure}
% first.py
%\centerline{\includegraphics[height=2.5in]{figs/first_agepreg_hist.pdf}}
\caption{Histogram of mother's age at end of pregnancy.}
\label{first_agepreg_hist}
\end{figure}

데이터를 읽고, 정상 출산에 대한 레코드를 선택해서 시작해보자.


\begin{verbatim}
    preg = nsfg.ReadFemPreg()
    live = preg[preg.outcome == 1]
\end{verbatim}

꺾쇠 표현식은 부울 시리즈(boolean Series)로 데이터프레임에서 행을 선택하고 새로운 데이터프레임을 반환한다. 다음에 정상출산에 대한 \verb"birthwgt_lb" 히스토그램을 생성하고 플롯해서 그려낸다.

\index{데이터프레임 (DataFrame)}
\index{시리즈 (Series)}
\index{Hist}
\index{꺾쇠 연산자 (bracket operator)}
\index{부울 (boolean)}

\begin{verbatim}
    hist = thinkstats2.Hist(live.birthwgt_lb, label='birthwgt_lb')
    thinkplot.Hist(hist)
    thinkplot.Show(xlabel='pounds', ylabel='frequency')
\end{verbatim}

Hist에 인자가 판다스 시리즈일 때, {\tt nan} 값은 탈락한다. 
{\tt label}은 문자열로 Hist가 그려질 때 범례에 나타난다. 

\index{판다스 (pandas)}
\index{시리즈 (Series)}
\index{thinkplot}
\index{NaN}

\begin{figure}
% first.py
%\centerline{\includegraphics[height=2.5in]{figs/first_prglngth_hist.pdf}}
\caption{Histogram of pregnancy length in weeks.}
\label{first_prglngth_hist}
\end{figure}

그림~\ref{first_wgt_lb_hist}가 결과를 보여준다.
가장 많이 관찰되는 값을 {\bf 최빈값(mode)}이라고 하고 이 경우에 7 파운드다. 
분포는 근사적으로 종모양인 {\bf 정규(normal)} 분포 모양으로 
{\bf 가우스(Gaussian)} 분포라고도 한다. 하지만, 순수 정규분포와 달리,
분포가 비대칭이다; 오른쪽보다 왼쪽으로 좀더 확장된 {\bf 꼬리(tail)}가 있다.


그림~\ref{first_wgt_oz_hist}은 변수 \verb"birthwgt_oz"의 히스토그램으로 출생 체중의 온스 부분이다. 
이론적으로 분포가 {\bf 균등(uniform)}하길 기대한다; 즉, 모든 값이 동일한 빈도를 가져야 한다.
사실 0 이 다른 값과 비교하여 좀더 흔하고, 1과 15는 좀더 흔하지 않은데, 아마도 이유는 응답자가 정수값에 가까운 출생 체중을 반올림한것으로 추측한다.

\index{출생 체중 (birth weight)}
\index{체중 (weight)!출생 (birth)}

그림~\ref{first_agepreg_hist}은 \verb"agepreg"의 히스토그램으로 임신 말기 산모 나이다.
분포는 대략 종모양이지만, 이 사례의 경우 꼬리가 좀더 왼쪽보다 오른쪽으로 뻗여나갔다. 
대부분의 산모는 20대이지만 30대도 적은 수지만 존재한다.

 
그림~\ref{first_prglngth_hist}은 \verb"prglngth"의 히스토그램으로 주간(week) 단위 임신기간이다.
가장 흔한 값은 39주다. 외쪽 꼬리가 오른쪽 꼬리보다 길다; 조산 신생아가 일반적이지만,
임신기간이 43주를 넘어가지는 않고, 만약 의사가 판단하기에 필요하다면 임신기간에 관여하는 것이 일반적이다.

\index{임신 기간 (pregnancy length)}


\section{특이점 (Outliers)}

히스토그램을 보면, 가장 흔한 값과 분포 모양을 식별하기는 쉽다. 하지만, 드문 값이 항상 눈에 잘 뜨이지는 않는다.
\index{히스토그램 (histogram)}

계속 진행하기 전에, {\bf 특이점 (outliers)}을 점검하는 것이 좋은 생각이다. 특이점은 이상치라고 불리는 극단값으로 측정과 기록에서 오류일 수도 있고, 드분 사건의 정확한 기록일 수도 있다.

\index{특이점 (outlier)}

Hist는 {\tt Largest}, {\tt Smallest} 메쏘드를 제공하는데, 정수 {\tt n}을 받아 히스토그램에서 최대값과 최소값 {\tt n}개를 반환한다.

\index{Hist}

\begin{verbatim}
    for weeks, freq in hist.Smallest(10):
        print(weeks, freq)
\end{verbatim}

정상 출산에 대한 임신기간 리스트에서 가장 작은 최소값 10개는 {\tt [0, 4, 9, 13, 17, 18, 19, 20, 21, 22]}이다. 10 주보다 적은 값은 확실한 오류다; 가장 그럴듯한 설명은 아마도 결과를 올바르게 코드화하지 못한 것이다. 30주 이상되는 값은 아마도 적합하다. 10주에서 30주 사이의 값은 확실하다고 하기가 어렵다; 몇몇 값은 아마도 오류지만 몇몇은 미숙아를 나타낸다.

\index{임신 기간 (pregnancy length)}

범위의 반대편에서 가장 큰 값은 다음과 같다.

%
\begin{verbatim}
weeks  count
43     148
44     46
45     10
46     1
47     1
48     7
50     2
\end{verbatim}

만약 임신기간이 42주를 넘어가면 대부분의 의사는 유도분만(induced labor)을 추천한다.
그래서 몇몇 긴 임신기간값은 놀라움을 준다. 특히, 임신 50주는 의학적으로 가능하지 않아 보인다.

특이점을 다루는 가장 좋은 방법은 ``특정 분야의 전문 지식(domain knowledge)''에 달려있다; 즉, 데이터 출처와 데이터가 의미하는 바에 대한 정보. 그리고 어떠한 분석방법을 수행할지에 달려있다.

\index{특이점 (outlier)}

이번 예제에서, 동기 부여 질문은 첫번째 아이가 일찍 (혹은 늦게) 태어나는 경향이 있느냐는 것이다.
사람들이 이 질문을 할 때, 대체로 전체 임신기간에 관심이 있다. 그래서, 이번 분석에서는 27주 이상이 되는 
임신에 초점을 맞출 것이다. 


\section{첫번째 아이 (First babies)}



Now we can compare the distribution of pregnancy lengths for first
babies and others.  I divided the DataFrame of live births using
{\tt birthord}, and computed their histograms:
\index{DataFrame}
\index{Hist}
\index{pregnancy length}

\begin{verbatim}
    firsts = live[live.birthord == 1]
    others = live[live.birthord != 1]

    first_hist = thinkstats2.Hist(firsts.prglngth)
    other_hist = thinkstats2.Hist(others.prglngth)
\end{verbatim}

Then I plotted their histograms on the same axis:

\begin{verbatim}
    width = 0.45
    thinkplot.PrePlot(2)
    thinkplot.Hist(first_hist, align='right', width=width)
    thinkplot.Hist(other_hist, align='left', width=width)
    thinkplot.Show(xlabel='weeks', ylabel='frequency')
\end{verbatim}

{\tt thinkplot.PrePlot} takes the number of histograms
we are planning to plot; it uses this information to choose
an appropriate collection of colors.
\index{thinkplot}

\begin{figure}
% first.py
%\centerline{\includegraphics[height=2.5in]{figs/first_nsfg_hist.pdf}}
\caption{Histogram of pregnancy lengths.}
\label{first_nsfg_hist}
\end{figure}

{\tt thinkplot.Hist} normally uses {\tt align='center'} so that
each bar is centered over its value.  For this figure, I use
{\tt align='right'} and {\tt align='left'} to place
corresponding bars on either side of the value.
\index{Hist}

With {\tt width=0.45}, the total width of the two bars is 0.9,
leaving some space between each pair.

Finally, I adjust the axis to show only data between 27 and 46 weeks.
Figure~\ref{first_nsfg_hist} shows the result.
\index{pregnancy length}
\index{length!pregnancy}

Histograms are useful because they make the most frequent values
immediately apparent.  But they are not the best choice for comparing
two distributions.  In this example, there are fewer ``first babies''
than ``others,'' so some of the apparent differences in the histograms
are due to sample sizes.  In the next chapter we address this problem
using probability mass functions.


\section{Summarizing distributions}
\label{mean}

A histogram is a complete description of the distribution of a sample;
that is, given a histogram, we could reconstruct the values in the
sample (although not their order).

If the details of the distribution are important, it might be
necessary to present a histogram.  But often we want to
summarize the distribution with a few descriptive statistics.

Some of the characteristics we might want to report are:

\begin{itemize}

\item central tendency: Do the values tend to cluster around
a particular point?
\index{central tendency}

\item modes: Is there more than one cluster?
\index{mode}

\item spread: How much variability is there in the values?
\index{spread}

\item tails: How quickly do the probabilities drop off as we
move away from the modes?
\index{tail}

\item outliers: Are there extreme values far from the modes?
\index{outlier}

\end{itemize}

Statistics designed to answer these questions are called {\bf summary
  statistics}.  By far the most common summary statistic is the {\bf
  mean}, which is meant to describe the central tendency of the
distribution.  \index{mean} \index{average} \index{summary statistic}

If you have a sample of {\tt n} values, $x_i$, the mean, $\xbar$, is
the sum of the values divided by the number of values; in other words
%
\[ \xbar = \frac{1}{n} \sum_i x_i \]
%
The words ``mean'' and ``average'' are sometimes used interchangeably,
but I make this distinction:

\begin{itemize}

\item The ``mean'' of a sample is the summary statistic computed with
  the previous formula.

\item An ``average'' is one of several summary statistics you might
  choose to describe a central tendency.
\index{central tendency}

\end{itemize}

Sometimes the mean is a good description of a set of values.  For
example, apples are all pretty much the same size (at least the ones
sold in supermarkets).  So if I buy 6 apples and the total weight is 3
pounds, it would be a reasonable summary to say they are about a half
pound each.
\index{weight!pumpkin}

But pumpkins are more diverse.  Suppose I grow several varieties in my
garden, and one day I harvest three decorative pumpkins that are 1
pound each, two pie pumpkins that are 3 pounds each, and one Atlantic
Giant\textregistered~pumpkin that weighs 591 pounds.  The mean of this
sample is 100 pounds, but if I told you ``The average pumpkin in my
garden is 100 pounds,'' that would be misleading.  In this example,
there is no meaningful average because there is no typical pumpkin.
\index{pumpkin}



\section{Variance}
\index{variance}

If there is no single number that summarizes pumpkin weights,
we can do a little better with two numbers: mean and {\bf variance}.

Variance is a summary statistic intended to describe the variability
or spread of a distribution.  The variance of a set of values is
%
\[ S^2 = \frac{1}{n} \sum_i (x_i - \xbar)^2 \]
%
The term $x_i - \xbar$ is called the ``deviation from the mean,'' so
variance is the mean squared deviation.  The square root of variance,
$S$, is the {\bf standard deviation}.  \index{deviation}
\index{standard deviation}
\index{deviation}

If you have prior experience, you might have seen a formula for
variance with $n-1$ in the denominator, rather than {\tt n}.  This
statistic is used to estimate the variance in a population using a
sample.  We will come back to this in Chapter~\ref{estimation}.
\index{sample variance}

Pandas data structures provides methods to compute mean, variance and
standard deviation:
\index{pandas}

\begin{verbatim}
    mean = live.prglngth.mean()
    var = live.prglngth.var()
    std = live.prglngth.std()
\end{verbatim}

For all live births, the mean pregnancy length is 38.6 weeks, the
standard deviation is 2.7 weeks, which means we should expect
deviations of 2-3 weeks to be common.
\index{pregnancy length}

Variance of pregnancy length is 7.3, which is hard to interpret,
especially since the units are weeks$^2$, or ``square weeks.''
Variance is useful in some calculations, but it is not
a good summary statistic.


\section{Effect size}
\index{effect size}

An {\bf effect size} is a summary statistic intended to describe (wait
for it) the size of an effect.  For example, to describe the
difference between two groups, one obvious choice is the difference in
the means.  \index{effect size}

Mean pregnancy length for first babies is 38.601; for
other babies it is 38.523.  The difference is 0.078 weeks, which works
out to 13 hours.  As a fraction of the typical pregnancy length, this
difference is about 0.2\%.
\index{pregnancy length}

If we assume this estimate is accurate, such a difference
would have no practical consequences.  In fact, without
observing a large number of pregnancies, it is unlikely that anyone
would notice this difference at all.
\index{effect size}

Another way to convey the size of the effect is to compare the
difference between groups to the variability within groups.
Cohen's $d$ is a statistic intended to do that; it is defined
%
\[ d = \frac{\bar{x_1} - \bar{x_2}}{s}  \]
%
where $\bar{x_1}$ and $\bar{x_2}$ are the means of the groups and
$s$ is the ``pooled standard deviation''.  Here's the Python
code that computes Cohen's $d$:
\index{standard deviation!pooled}

\begin{verbatim}
def CohenEffectSize(group1, group2):
    diff = group1.mean() - group2.mean()

    var1 = group1.var()
    var2 = group2.var()
    n1, n2 = len(group1), len(group2)

    pooled_var = (n1 * var1 + n2 * var2) / (n1 + n2)
    d = diff / math.sqrt(pooled_var)
    return d
\end{verbatim}

In this example, the difference in means is 0.029 standard deviations,
which is small.  To put that in perspective, the difference in
height between men and women is about 1.7 standard deviations (see
\url{https://en.wikipedia.org/wiki/Effect_size}).


\section{Reporting results}

We have seen several ways to describe the difference in pregnancy
length (if there is one) between first babies and others.  How should
we report these results?
\index{pregnancy length}

The answer depends on who is asking the question.  A scientist might
be interested in any (real) effect, no matter how small.  A doctor
might only care about effects that are {\bf clinically significant};
that is, differences that affect treatment decisions.  A pregnant
woman might be interested in results that are relevant to her, like
the probability of delivering early or late.
\index{clinically significant} \index{significant}

How you report results also depends on your goals.  If you are trying
to demonstrate the importance of an effect, you might choose summary
statistics that emphasize differences.  If you are trying to reassure
a patient, you might choose statistics that put the differences in
context.

Of course your decisions should also be guided by professional ethics.
It's ok to be persuasive; you {\em should} design statistical reports
and visualizations that tell a story clearly.  But you should also do
your best to make your reports honest, and to acknowledge uncertainty
and limitations.
\index{ethics}


\section{Exercises}

\begin{exercise}
Based on the results in this chapter, suppose you were asked to
summarize what you learned about whether first babies arrive late.

Which summary statistics would you use if you wanted to get a story
on the evening news?  Which ones would you use if you wanted to
reassure an anxious patient?
\index{Adams, Cecil}
\index{Straight Dope, The}

Finally, imagine that you are Cecil Adams, author of {\it The Straight
  Dope} (\url{http://straightdope.com}), and your job is to answer the
question, ``Do first babies arrive late?''  Write a paragraph that
uses the results in this chapter to answer the question clearly,
precisely, and honestly.
\index{ethics}

\end{exercise}

\begin{exercise}
In the repository you downloaded, you should find a file named
\verb"chap02ex.ipynb"; open it.  Some cells are already filled in, and
you should execute them.  Other cells give you instructions for
exercises.  Follow the instructions and fill in the answers.

A solution to this exercise is in \verb"chap02soln.ipynb"
\end{exercise}

For the following exercises, create a file named {\tt chap02ex.py}.
You can find a solution in \verb"chap02soln.py".

\begin{exercise}
The mode of a distribution is the most frequent value; see
\url{http://wikipedia.org/wiki/Mode_(statistics)}.  Write a function
called {\tt Mode} that takes a Hist and returns the most
frequent value.\index{mode}
\index{Hist}

As a more challenging exercise, write a function called {\tt AllModes}
that returns a list of value-frequency pairs in descending order of
frequency.
\index{frequency}
\end{exercise}

\begin{exercise}
Using the variable \verb"totalwgt_lb", investigate whether first
babies are lighter or heavier than others.  Compute Cohen's $d$
to quantify the difference between the groups.  How does it
compare to the difference in pregnancy length?
\index{pregnancy length}
\end{exercise}


\section{Glossary}

\begin{itemize}

\item distribution: The values that appear in a sample
and the frequency of each.
\index{distribution}

\item histogram: A mapping from values to frequencies, or a graph
that shows this mapping.
\index{histogram}

\item frequency: The number of times a value appears in a sample.
\index{frequency}

\item mode: The most frequent value in a sample, or one of the
most frequent values.
\index{mode}

\item normal distribution: An idealization of a bell-shaped distribution;
also known as a Gaussian distribution. 
\index{Gaussian distribution}
\index{normal distribution}

\item uniform distribution: A distribution in which all values have
the same frequency.
\index{uniform distribution}

\item tail: The part of a distribution at the high and low extremes.
\index{tail}

\item central tendency: A characteristic of a sample or population;
intuitively, it is an average or typical value. 
\index{central tendency}

\item outlier: A value far from the central tendency.
\index{outlier}

\item spread: A measure of how spread out the values in a distribution
are.
\index{spread}

\item summary statistic: A statistic that quantifies some aspect
of a distribution, like central tendency or spread.
\index{summary statistic}

\item variance: A summary statistic often used to quantify spread.
\index{variance}

\item standard deviation: The square root of variance, also used
as a measure of spread.
\index{standard deviation}

\item effect size: A summary statistic intended to quantify the size
of an effect like a difference between groups.
\index{effect size}

\item clinically significant: A result, like a difference between groups,
that is relevant in practice.
\index{clinically significant}

\end{itemize}



