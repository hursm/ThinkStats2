\chapter{Probability mass functions}
\index{probability mass function}

The code for this chapter is in {\tt probability.py}.
For information about downloading and
working with this code, see Section~\ref{code}.


\section{Pmfs}
\index{Pmf}

Another way to represent a distribution is a {\bf probability mass
  function} (PMF), which maps from each value to its probability.  A
{\bf probability} is a frequency expressed as a fraction of the sample
size, {\tt n}.  To get from frequencies to probabilities, we divide
through by {\tt n}, which is called {\bf normalization}.
\index{frequency}
\index{probability}
\index{normalization}
\index{PMF}
\index{probability mass function}

Given a Hist, we can make a dictionary that maps from each
value to its probability: \index{Hist}
%
\begin{verbatim}
n = hist.Total()
d = {}
for x, freq in hist.Items():
    d[x] = freq / n
\end{verbatim}
%
Or we can use the Pmf class provided by {\tt thinkstats2}.
Like Hist, the Pmf constructor can take a list, pandas
Series, dictionary, Hist, or another Pmf object.  Here's an example
with a simple list:
%
\begin{verbatim}
>>> import thinkstats2
>>> pmf = thinkstats2.Pmf([1, 2, 2, 3, 5])
>>> pmf
Pmf({1: 0.2, 2: 0.4, 3: 0.2, 5: 0.2})
\end{verbatim}

The Pmf is normalized so total probability is 1.

Pmf and Hist objects are similar in many ways; in fact, they inherit
many of their methods from a common parent class.  For example, the
methods {\tt Values} and {\tt Items} work the same way for both.  The
biggest difference is that a Hist maps from values to integer
counters; a Pmf maps from values to floating-point probabilities.
\index{Hist}

To look up the probability associated with a value, use {\tt Prob}:
%
\begin{verbatim}
>>> pmf.Prob(2)
0.4
\end{verbatim}

The bracket operator is equivalent:
\index{bracket operator}

\begin{verbatim}
>>> pmf[2]
0.4
\end{verbatim}

You can modify an existing Pmf by incrementing the probability
associated with a value:
%
\begin{verbatim}
>>> pmf.Incr(2, 0.2)
>>> pmf.Prob(2)
0.6
\end{verbatim}

Or you can multiply a probability by a factor:
%
\begin{verbatim}
>>> pmf.Mult(2, 0.5)
>>> pmf.Prob(2)
0.3
\end{verbatim}

If you modify a Pmf, the result may not be normalized; that is, the
probabilities may no longer add up to 1.  To check, you can call {\tt
  Total}, which returns the sum of the probabilities:
%
\begin{verbatim}
>>> pmf.Total()
0.9
\end{verbatim}

To renormalize, call {\tt Normalize}:
%
\begin{verbatim}
>>> pmf.Normalize()
>>> pmf.Total()
1.0
\end{verbatim}

Pmf objects provide a {\tt Copy} method so you can make
and modify a copy without affecting the original.
\index{Pmf}

My notation in this section might seem inconsistent, but there is a
system: I use Pmf for the name of the class, {\tt pmf} for an instance
of the class, and PMF for the mathematical concept of a
probability mass function.


\section{Plotting PMFs}
\index{PMF}

{\tt thinkplot} provides two ways to plot Pmfs:
\index{thinkplot}

\begin{itemize}

\item To plot a Pmf as a bar graph, you can use 
{\tt thinkplot.Hist}.  Bar graphs are most useful if the number
of values in the Pmf is small.
\index{bar plot}
\index{plot!bar}

\item To plot a Pmf as a step function, you can use
{\tt thinkplot.Pmf}.  This option is most useful if there are
a large number of values and the Pmf is smooth.  This function
also works with Hist objects.
\index{line plot}
\index{plot!line}
\index{Hist}
\index{Pmf}

\end{itemize}

In addition, {\tt pyplot} provides a function called {\tt hist} that
takes a sequence of values, computes a histogram, and plots it.
Since I use Hist objects, I usually don't use {\tt pyplot.hist}.
\index{pyplot}

\begin{figure}
% probability.py
%\centerline{\includegraphics[height=3.0in]{figs/probability_nsfg_pmf.pdf}}
\caption{PMF of pregnancy lengths for first babies and others, using
  bar graphs and step functions.}
\label{probability_nsfg_pmf}
\end{figure}
\index{pregnancy length}
\index{length!pregnancy}

Figure~\ref{probability_nsfg_pmf} shows PMFs of pregnancy length for
first babies and others using bar graphs (left) and step functions
(right).
\index{pregnancy length}

By plotting the PMF instead of the histogram, we can compare the two
distributions without being mislead by the difference in sample
size.  Based on this figure, first babies seem to be less likely than
others to arrive on time (week 39) and more likely to be a late (weeks
41 and 42).

Here's the code that generates Figure~\ref{probability_nsfg_pmf}:

\begin{verbatim}
    thinkplot.PrePlot(2, cols=2)
    thinkplot.Hist(first_pmf, align='right', width=width)
    thinkplot.Hist(other_pmf, align='left', width=width)
    thinkplot.Config(xlabel='weeks',
                     ylabel='probability',
                     axis=[27, 46, 0, 0.6])

    thinkplot.PrePlot(2)
    thinkplot.SubPlot(2)
    thinkplot.Pmfs([first_pmf, other_pmf])
    thinkplot.Show(xlabel='weeks',
                   axis=[27, 46, 0, 0.6])
\end{verbatim}

{\tt PrePlot} takes optional parameters {\tt rows} and {\tt cols}
to make a grid of figures, in this case one row of two figures.
The first figure (on the left) displays the Pmfs using {\tt thinkplot.Hist},
as we have seen before.
\index{thinkplot}
\index{Hist}

The second call to {\tt PrePlot} resets the color generator.  Then
{\tt SubPlot} switches to the second figure (on the right) and
displays the Pmfs using {\tt thinkplot.Pmfs}.  I used the {\tt axis} option
to ensure that the two figures are on the same axes, which is
generally a good idea if you intend to compare two figures.


\section{Other visualizations}
\label{visualization}

Histograms and PMFs are useful while you are exploring data and
trying to identify patterns and relationships.
Once you have an idea what is going on, a good next step is to
design a visualization that makes the patterns you have identified
as clear as possible.
\index{exploratory data analysis}
\index{visualization}

In the NSFG data, the biggest differences in the distributions are
near the mode.  So it makes sense to zoom in on that part of the
graph, and to transform the data to emphasize differences:
\index{National Survey of Family Growth}
\index{NSFG}

\begin{verbatim}
    weeks = range(35, 46)
    diffs = []
    for week in weeks:
        p1 = first_pmf.Prob(week)
        p2 = other_pmf.Prob(week)
        diff = 100 * (p1 - p2)
        diffs.append(diff)

    thinkplot.Bar(weeks, diffs)
\end{verbatim}

In this code, {\tt weeks} is the range of weeks; {\tt diffs} is the
difference between the two PMFs in percentage points.
Figure~\ref{probability_nsfg_diffs} shows the result as a bar chart.
This figure makes the pattern clearer: first babies are less likely to
be born in week 39, and somewhat more likely to be born in weeks 41
and 42.
\index{thinkplot}

\begin{figure}
% probability.py
%\centerline{\includegraphics[height=2.5in]{figs/probability_nsfg_diffs.pdf}}
\caption{Difference, in percentage points, by week.}
\label{probability_nsfg_diffs}
\end{figure}

For now we should hold this conclusion only tentatively.
We used the same dataset to identify an
apparent difference and then chose a visualization that makes the
difference apparent.  We can't be sure this effect is real;
it might be due to random variation.  We'll address this concern
later.


\section{The class size paradox}
\index{class size}

Before we go on, I want to demonstrate
one kind of computation you can do with Pmf objects; I call
this example the ``class size paradox.''
\index{Pmf}

At many American colleges and universities, the student-to-faculty
ratio is about 10:1.  But students are often surprised to discover
that their average class size is bigger than 10.  There
are two reasons for the discrepancy:

\begin{itemize}

\item Students typically take 4--5 classes per semester, but
professors often teach 1 or 2.

\item The number of students who enjoy a small class is small,
but the number of students in a large class is (ahem!) large.

\end{itemize}

The first effect is obvious, at least once it is pointed out;
the second is more subtle.  Let's look at an example.  Suppose
that a college offers 65 classes in a given semester, with the
following distribution of sizes:
%
\begin{verbatim}
 size      count
 5- 9          8
10-14          8
15-19         14
20-24          4
25-29          6
30-34         12
35-39          8
40-44          3
45-49          2
\end{verbatim}

If you ask the Dean for the average class size, he would
construct a PMF, compute the mean, and report that the
average class size is 23.7.  Here's the code:

\begin{verbatim}
    d = { 7: 8, 12: 8, 17: 14, 22: 4, 
          27: 6, 32: 12, 37: 8, 42: 3, 47: 2 }

    pmf = thinkstats2.Pmf(d, label='actual')
    print('mean', pmf.Mean())
\end{verbatim}

But if you survey a group of students, ask them how many
students are in their classes, and compute the mean, you would
think the average class was bigger.  Let's see how
much bigger.

First, I compute the
distribution as observed by students, where the probability
associated with each class size is ``biased'' by the number
of students in the class.
\index{observer bias}
\index{bias!observer}

\begin{verbatim}
def BiasPmf(pmf, label):
    new_pmf = pmf.Copy(label=label)

    for x, p in pmf.Items():
        new_pmf.Mult(x, x)
        
    new_pmf.Normalize()
    return new_pmf
\end{verbatim}

For each class size, {\tt x}, we multiply the probability by
{\tt x}, the number of students who observe that class size.
The result is a new Pmf that represents the biased distribution.

Now we can plot the actual and observed distributions:
\index{thinkplot}

\begin{verbatim}
    biased_pmf = BiasPmf(pmf, label='observed')
    thinkplot.PrePlot(2)
    thinkplot.Pmfs([pmf, biased_pmf])
    thinkplot.Show(xlabel='class size', ylabel='PMF')
\end{verbatim}

\begin{figure}
% probability.py
%\centerline{\includegraphics[height=3.0in]{figs/class_size1.pdf}}
\caption{Distribution of class sizes, actual and as observed by students.}
\label{class_size1}
\end{figure}

Figure~\ref{class_size1} shows the result.  In the biased distribution
there are fewer small classes and more large ones.
The mean of the biased distribution is 29.1, almost 25\% higher
than the actual mean.

It is also possible to invert this operation.  Suppose you want to
find the distribution of class sizes at a college, but you can't get
reliable data from the Dean.  An alternative is to choose a random
sample of students and ask how many students are in their
classes.  \index{bias!oversampling} \index{oversampling}

The result would be biased for the reasons we've just seen, but you
can use it to estimate the actual distribution.  Here's the function
that unbiases a Pmf:

\begin{verbatim}
def UnbiasPmf(pmf, label):
    new_pmf = pmf.Copy(label=label)

    for x, p in pmf.Items():
        new_pmf.Mult(x, 1.0/x)
        
    new_pmf.Normalize()
    return new_pmf
\end{verbatim}

It's similar to {\tt BiasPmf}; the only difference is that it
divides each probability by {\tt x} instead of multiplying.


\section{DataFrame indexing}

In Section~\ref{dataframe} we read a pandas DataFrame and used it to
select and modify data columns.  Now let's look at row selection.
To start, I create a NumPy array of random numbers and use it
to initialize a DataFrame:
\index{NumPy}
\index{pandas}
\index{DataFrame}

\begin{verbatim}
>>> import numpy as np
>>> import pandas
>>> array = np.random.randn(4, 2)
>>> df = pandas.DataFrame(array)
>>> df
          0         1
0 -0.143510  0.616050
1 -1.489647  0.300774
2 -0.074350  0.039621
3 -1.369968  0.545897
\end{verbatim}

By default, the rows and columns are numbered starting at zero, but
you can provide column names:

\begin{verbatim}
>>> columns = ['A', 'B']
>>> df = pandas.DataFrame(array, columns=columns)
>>> df
          A         B
0 -0.143510  0.616050
1 -1.489647  0.300774
2 -0.074350  0.039621
3 -1.369968  0.545897
\end{verbatim}

You can also provide row names.  The set of row names is called the
{\bf index}; the row names themselves are called {\bf labels}.

\begin{verbatim}
>>> index = ['a', 'b', 'c', 'd']
>>> df = pandas.DataFrame(array, columns=columns, index=index)
>>> df
          A         B
a -0.143510  0.616050
b -1.489647  0.300774
c -0.074350  0.039621
d -1.369968  0.545897
\end{verbatim}

As we saw in the previous chapter, simple indexing selects a
column, returning a Series:
\index{Series}

\begin{verbatim}
>>> df['A']
a   -0.143510
b   -1.489647
c   -0.074350
d   -1.369968
Name: A, dtype: float64
\end{verbatim}

To select a row by label, you can use the {\tt loc} attribute, which
returns a Series:

\begin{verbatim}
>>> df.loc['a']
A   -0.14351
B    0.61605
Name: a, dtype: float64
\end{verbatim}

If you know the integer position of a row, rather than its label, you
can use the {\tt iloc} attribute, which also returns a Series.

\begin{verbatim}
>>> df.iloc[0]
A   -0.14351
B    0.61605
Name: a, dtype: float64
\end{verbatim}

{\tt loc} can also take a list of labels; in that case,
the result is a DataFrame.

\begin{verbatim}
>>> indices = ['a', 'c']
>>> df.loc[indices]
         A         B
a -0.14351  0.616050
c -0.07435  0.039621
\end{verbatim}

Finally, you can use a slice to select a range of rows by label:

\begin{verbatim}
>>> df['a':'c']
          A         B
a -0.143510  0.616050
b -1.489647  0.300774
c -0.074350  0.039621
\end{verbatim}

Or by integer position:

\begin{verbatim}
>>> df[0:2]
          A         B
a -0.143510  0.616050
b -1.489647  0.300774
\end{verbatim}

The result in either case is a DataFrame, but notice that the first
result includes the end of the slice; the second doesn't.
\index{DataFrame}

My advice: if your rows have labels that are not simple integers, use
the labels consistently and avoid using integer positions.



\section{Exercises}

Solutions to these exercises are in \verb"chap03soln.ipynb"
and \verb"chap03soln.py"

\begin{exercise}
Something like the class size paradox appears if you survey children
and ask how many children are in their family.  Families with many
children are more likely to appear in your sample, and
families with no children have no chance to be in the sample.
\index{observer bias}
\index{bias!observer}

Use the NSFG respondent variable \verb"NUMKDHH" to construct the actual
distribution for the number of children under 18 in the household.

Now compute the biased distribution we would see if we surveyed the
children and asked them how many children under 18 (including themselves)
are in their household.  

Plot the actual and biased distributions, and compute their means.
As a starting place, you can use \verb"chap03ex.ipynb".
\end{exercise}


\begin{exercise}
\index{mean}
\index{variance}
\index{PMF}

In Section~\ref{mean} we computed the mean of a sample by adding up
the elements and dividing by n.  If you are given a PMF, you can
still compute the mean, but the process is slightly different:
%
\[ \xbar = \sum_i p_i~x_i \]
%
where the $x_i$ are the unique values in the PMF and $p_i=PMF(x_i)$.
Similarly, you can compute variance like this:
%
\[ S^2 = \sum_i p_i~(x_i - \xbar)^2\]
% 
Write functions called {\tt PmfMean} and {\tt PmfVar} that take a
Pmf object and compute the mean and variance.  To test these methods,
check that they are consistent with the methods {\tt Mean} and {\tt
  Var} provided by Pmf.
\index{Pmf}

\end{exercise}


\begin{exercise}
I started with the question, ``Are first babies more likely
to be late?''  To address it, I computed the difference in
means between groups of babies, but I ignored the possibility
that there might be a difference between first babies and
others {\em for the same woman}.

To address this version of the question, select respondents who
have at least two babies and compute pairwise differences.  Does
this formulation of the question yield a different result?

Hint: use {\tt nsfg.MakePregMap}.
\end{exercise}


\begin{exercise}
\label{relay}

In most foot races, everyone starts at the same time.  If you are a
fast runner, you usually pass a lot of people at the beginning of the
race, but after a few miles everyone around you is going at the same
speed.
\index{relay race}

When I ran a long-distance (209 miles) relay race for the first
time, I noticed an odd phenomenon: when I overtook another runner, I
was usually much faster, and when another runner overtook me, he was
usually much faster.

At first I thought that the distribution of speeds might be bimodal;
that is, there were many slow runners and many fast runners, but few
at my speed.

Then I realized that I was the victim of a bias similar to the
effect of class size.  The race
was unusual in two ways: it used a staggered start, so teams started
at different times; also, many teams included runners at different
levels of ability. \index{bias!selection} \index{selection bias}

As a result, runners were spread out along the course with little
relationship between speed and location.  When I joined the race, the
runners near me were (pretty much) a random sample of the runners in
the race.

So where does the bias come from?  During my time on the course, the
chance of overtaking a runner, or being overtaken, is proportional to
the difference in our speeds.  I am more likely to catch a slow
runner, and more likely to be caught by a fast runner.  But runners
at the same speed are unlikely to see each other.

Write a function called {\tt ObservedPmf} that takes a Pmf representing
the actual distribution of runners' speeds, and the speed of a running
observer, and returns a new Pmf representing the distribution of
runners' speeds as seen by the observer.
\index{observer bias}
\index{bias!observer}

To test your function, you can use {\tt relay.py}, which  reads the
results from the James Joyce Ramble 10K in Dedham MA and converts the
pace of each runner to mph.

Compute the distribution of speeds you would observe if you ran a
relay race at 7.5 mph with this group of runners.  A solution to this
exercise is in \verb"relay_soln.py".
\end{exercise}


\section{Glossary}

\begin{itemize}

\item Probability mass function (PMF): a representation of a distribution
as a function that maps from values to probabilities.
\index{PMF}
\index{probability mass function}

\item probability: A frequency expressed as a fraction of the sample
size.
\index{frequency}
\index{probability}

\item normalization: The process of dividing a frequency by a sample
size to get a probability.
\index{normalization}

\item index: In a pandas DataFrame, the index is a special column
that contains the row labels.
\index{pandas}
\index{DataFrame}

\end{itemize}
