
\chapter{Cumulative distribution functions}
\label{cumulative}

The code for this chapter is in {\tt cumulative.py}.
For information about downloading and
working with this code, see Section~\ref{code}.


\section{The limits of PMFs}
\index{PMF}

PMFs work well if the number of values is small.  But as the number of
values increases, the probability associated with each value gets
smaller and the effect of random noise increases.

For example, we might be interested in the distribution of birth
weights.  In the NSFG data, the variable \verb"totalwgt_lb" records
weight at birth in pounds.  Figure~\ref{nsfg_birthwgt_pmf} shows
the PMF of these values for first babies and others.
\index{National Survey of Family Growth} \index{NSFG} \index{birth weight}
\index{weight!birth}

\begin{figure}
% cumulative.py
%\centerline{\includegraphics[height=2.5in]{figs/nsfg_birthwgt_pmf.pdf}}
\caption{PMF of birth weights.  This figure shows a limitation
of PMFs: they are hard to compare visually.}
\label{nsfg_birthwgt_pmf}
\end{figure}

Overall, these distributions resemble the bell shape of a normal
distribution, with many values near the mean and a few values much
higher and lower.

But parts of this figure are hard to interpret.  There are many spikes
and valleys, and some apparent differences between the distributions.
It is hard to tell which of these features are meaningful.  Also, it
is hard to see overall patterns; for example, which distribution do
you think has the higher mean?
\index{binning}

These problems can be mitigated by binning the data; that is, dividing
the range of values into non-overlapping intervals and counting the
number of values in each bin.  Binning can be useful, but it is tricky
to get the size of the bins right.  If they are big enough to smooth
out noise, they might also smooth out useful information.

An alternative that avoids these problems is the cumulative
distribution function (CDF), which is the subject of this chapter.
But before I can explain CDFs, I have to explain percentiles.
\index{CDF}


\section{Percentiles}
\index{percentile rank}

If you have taken a standardized test, you probably got your
results in the form of a raw score and a {\bf percentile rank}.
In this context, the percentile rank is the fraction of people who
scored lower than you (or the same).  So if you are ``in the 90th
percentile,'' you did as well as or better than 90\% of the people who
took the exam.

Here's how you could compute the percentile rank of a value,
\verb"your_score", relative to the values in the sequence {\tt
  scores}:
%
\begin{verbatim}
def PercentileRank(scores, your_score):
    count = 0
    for score in scores:
        if score <= your_score:
            count += 1

    percentile_rank = 100.0 * count / len(scores)
    return percentile_rank
\end{verbatim}

As an example, if the
scores in the sequence were 55, 66, 77, 88 and 99, and you got the 88,
then your percentile rank would be {\tt 100 * 4 / 5} which is 80.

If you are given a value, it is easy to find its percentile rank; going
the other way is slightly harder.  If you are given a percentile rank
and you want to find the corresponding value, one option is to
sort the values and search for the one you want:
%
\begin{verbatim}
def Percentile(scores, percentile_rank):
    scores.sort()
    for score in scores:
        if PercentileRank(scores, score) >= percentile_rank:
            return score
\end{verbatim}

The result of this calculation is a {\bf percentile}.  For example,
the 50th percentile is the value with percentile rank 50.  In the
distribution of exam scores, the 50th percentile is 77.
\index{percentile}

This implementation of {\tt Percentile} is not efficient.  A
better approach is to use the percentile rank to compute the index of
the corresponding percentile:

\begin{verbatim}
def Percentile2(scores, percentile_rank):
    scores.sort()
    index = percentile_rank * (len(scores)-1) // 100
    return scores[index]
\end{verbatim}

The difference between ``percentile'' and ``percentile rank'' can
be confusing, and people do not always use the terms precisely.
To summarize, {\tt PercentileRank} takes a value and computes
its percentile rank in a set of values; {\tt Percentile} takes
a percentile rank and computes the corresponding value.
\index{percentile rank}


\section{CDFs}
\index{CDF}

Now that we understand percentiles and percentile ranks,
we are ready to tackle the {\bf cumulative distribution function}
(CDF).  The CDF is the function that maps from a value to its
percentile rank.
\index{cumulative distribution function}
\index{percentile rank}

The CDF is a function of $x$, where $x$ is any value that might appear
in the distribution.  To evaluate $\CDF(x)$ for a particular value of
$x$, we compute the fraction of values in the distribution less
than or equal to $x$.

Here's what that looks like as a function that takes a sequence,
{\tt sample}, and a value, {\tt x}:
%
\begin{verbatim}
def EvalCdf(sample, x):
    count = 0.0
    for value in sample:
        if value <= x:
            count += 1

    prob = count / len(sample)
    return prob
\end{verbatim}

This function is almost identical to {\tt PercentileRank}, except that
the result is a probability in the range 0--1 rather than a
percentile rank in the range 0--100.
\index{sample}

As an example, suppose we collect a sample with the values 
{\tt [1, 2, 2, 3, 5]}.  Here are some values from its CDF:
%
\[ CDF(0) = 0 \]
%
\[ CDF(1) = 0.2\]
%
\[ CDF(2) = 0.6\]
%
\[ CDF(3) = 0.8\]
%
\[ CDF(4) = 0.8\]
%
\[ CDF(5) = 1\]
%
We can evaluate the CDF for any value of $x$, not just
values that appear in the sample.
If $x$ is less than the smallest value in the sample, $\CDF(x)$ is 0.
If $x$ is greater than the largest value, $\CDF(x)$ is 1.

\begin{figure}
% cumulative.py
%\centerline{\includegraphics[height=2.5in]{figs/cumulative_example_cdf.pdf}}
\caption{Example of a CDF.}
\label{example_cdf}
\end{figure}

Figure~\ref{example_cdf} is a graphical representation of this CDF.
The CDF of a sample is a step function.
\index{step function}


\section{Representing CDFs}
\index{Cdf}

{\tt thinkstats2} provides a class named Cdf that represents
CDFs.  The fundamental methods Cdf provides are:

\begin{itemize}

\item {\tt Prob(x)}: Given a value {\tt x}, computes the probability
  $p = \CDF(x)$.  The bracket operator is equivalent to {\tt Prob}.
\index{bracket operator}

\item {\tt Value(p)}: Given a probability {\tt p}, computes the
corresponding value, {\tt x}; that is, the {\bf inverse CDF} of {\tt p}.
\index{inverse CDF}
\index{CDF, inverse}

\end{itemize}

\begin{figure}
% cumulative.py
%\centerline{\includegraphics[height=2.5in]{figs/cumulative_prglngth_cdf.pdf}}
\caption{CDF of pregnancy length.}
\label{cumulative_prglngth_cdf}
\end{figure}

The Cdf constructor can take as an argument a list of values,
a pandas Series, a Hist, Pmf, or another Cdf.  The following
code makes a Cdf for the distribution of pregnancy lengths in
the NSFG:
\index{NSFG}
\index{pregnancy length}

\begin{verbatim}
    live, firsts, others = first.MakeFrames()
    cdf = thinkstats2.Cdf(live.prglngth, label='prglngth')
\end{verbatim}

{\tt thinkplot} provides a function named {\tt Cdf} that
plots Cdfs as lines:
\index{thinkplot}

\begin{verbatim}
    thinkplot.Cdf(cdf)
    thinkplot.Show(xlabel='weeks', ylabel='CDF')
\end{verbatim}

Figure~\ref{cumulative_prglngth_cdf} shows the result.  One way to
read a CDF is to look up percentiles.  For example, it looks like
about 10\% of pregnancies are shorter than 36 weeks, and about 90\%
are shorter than 41 weeks.  The CDF also provides a visual
representation of the shape of the distribution.  Common values appear
as steep or vertical sections of the CDF; in this example, the mode at
39 weeks is apparent.  There are few values below 30 weeks, so
the CDF in this range is flat.
\index{CDF, interpreting}

It takes some time to get used to CDFs, but once you
do, I think you will find that they show more information, more
clearly, than PMFs.


\section{Comparing CDFs}
\label{birth_weights}
\index{National Survey of Family Growth}
\index{NSFG}
\index{birth weight}
\index{weight!birth}

CDFs are especially useful for comparing distributions.  For
example, here is the code that plots the CDF of birth
weight for first babies and others.
\index{thinkplot}
\index{distributions, comparing}

\begin{verbatim}
    first_cdf = thinkstats2.Cdf(firsts.totalwgt_lb, label='first')
    other_cdf = thinkstats2.Cdf(others.totalwgt_lb, label='other')

    thinkplot.PrePlot(2)
    thinkplot.Cdfs([first_cdf, other_cdf])
    thinkplot.Show(xlabel='weight (pounds)', ylabel='CDF')
\end{verbatim}

\begin{figure}
% cumulative.py
%\centerline{\includegraphics[height=2.5in]{figs/cumulative_birthwgt_cdf.pdf}}
\caption{CDF of birth weights for first babies and others.}
\label{cumulative_birthwgt_cdf}
\end{figure}

Figure~\ref{cumulative_birthwgt_cdf} shows the result.
Compared to Figure~\ref{nsfg_birthwgt_pmf},
this figure makes the shape of the distributions, and the differences
between them, much clearer.  We can see that first babies are slightly
lighter throughout the distribution, with a larger discrepancy above 
the mean.
\index{shape}




\section{Percentile-based statistics}
\index{summary statistic}
\index{interquartile range}
\index{quartile}
\index{percentile}
\index{median}
\index{central tendency}
\index{spread}

Once you have computed a CDF, it is easy to compute percentiles
and percentile ranks.  The Cdf class provides these two methods:
\index{Cdf}
\index{percentile rank}

\begin{itemize}

\item {\tt PercentileRank(x)}: Given a value {\tt x}, computes its
  percentile rank, $100 \cdot \CDF(x)$.

\item {\tt Percentile(p)}: Given a percentile rank {\tt rank},
  computes the corresponding value, {\tt x}.  Equivalent to {\tt
    Value(p/100)}.

\end{itemize}

{\tt Percentile} can be used to compute percentile-based summary
statistics.  For example, the 50th percentile is the value that
divides the distribution in half, also known as the {\bf median}.
Like the mean, the median is a measure of the central tendency
of a distribution.

Actually, there are several definitions of ``median,'' each with
different properties.  But {\tt Percentile(50)} is simple and
efficient to compute.

Another percentile-based statistic is the {\bf interquartile range} (IQR),
which is a measure of the spread of a distribution.  The IQR
is the difference between the 75th and 25th percentiles.

More generally, percentiles are often used to summarize the shape
of a distribution.  For example, the distribution of income is
often reported in ``quintiles''; that is, it is split at the
20th, 40th, 60th and 80th percentiles.  Other distributions
are divided into ten ``deciles''.  Statistics like these that represent
equally-spaced points in a CDF are called {\bf quantiles}.
For more, see \url{https://en.wikipedia.org/wiki/Quantile}.
\index{quantile}
\index{quintile}
\index{decile}



\section{Random numbers}
\label{random}
\index{random number}

Suppose we choose a random sample from the population of live
births and look up the percentile rank of their birth weights.
Now suppose we compute the CDF of the percentile ranks.  What do
you think the distribution will look like?
\index{percentile rank}
\index{birth weight}
\index{weight!birth}

Here's how we can compute it.  First, we make the Cdf of
birth weights:
\index{Cdf}

\begin{verbatim}
    weights = live.totalwgt_lb
    cdf = thinkstats2.Cdf(weights, label='totalwgt_lb')
\end{verbatim}

Then we generate a sample and compute the percentile rank of
each value in the sample.

\begin{verbatim}
    sample = np.random.choice(weights, 100, replace=True)
    ranks = [cdf.PercentileRank(x) for x in sample]
\end{verbatim}

{\tt sample}
is a random sample of 100 birth weights, chosen with {\bf replacement};
that is, the same value could be chosen more than once.  {\tt ranks}
is a list of percentile ranks.
\index{replacement}

Finally we make and plot the Cdf of the percentile ranks.
\index{thinkplot}

\begin{verbatim}
    rank_cdf = thinkstats2.Cdf(ranks)
    thinkplot.Cdf(rank_cdf)
    thinkplot.Show(xlabel='percentile rank', ylabel='CDF')
\end{verbatim}

\begin{figure}
% cumulative.py
%\centerline{\includegraphics[height=2.5in]{figs/cumulative_random.pdf}}
\caption{CDF of percentile ranks for a random sample of birth weights.}
\label{cumulative_random}
\end{figure}

Figure~\ref{cumulative_random} shows the result.  The CDF is
approximately a straight line, which means that the distribution
is uniform.

That outcome might be non-obvious, but it is a consequence of
the way the CDF is defined.  What this figure shows is that 10\%
of the sample is below the 10th percentile, 20\% is below the
20th percentile, and so on, exactly as we should expect.

So, regardless of the shape of the CDF, the distribution of
percentile ranks is uniform.  This property is useful, because it
is the basis of a simple and efficient algorithm for generating
random numbers with a given CDF.  Here's how:
\index{inverse CDF algorithm}
\index{random number}

\begin{itemize}

\item Choose a percentile rank uniformly from the range 0--100.

\item Use {\tt Cdf.Percentile} to find the value in the distribution
that corresponds to the percentile rank you chose.
\index{Cdf}

\end{itemize}

Cdf provides an implementation of this algorithm, called
{\tt Random}:

\begin{verbatim}
# class Cdf:
    def Random(self):
        return self.Percentile(random.uniform(0, 100))
\end{verbatim}

Cdf also provides {\tt Sample}, which takes an integer,
{\tt n}, and returns a list of {\tt n} values chosen at random
from the Cdf.


\section{Comparing percentile ranks}

Percentile ranks are useful for comparing measurements across
different groups.  For example, people who compete in foot races are
usually grouped by age and gender.  To compare people in different
age groups, you can convert race times to percentile ranks.
\index{percentile rank}

A few years ago I ran the James Joyce Ramble 10K in
Dedham MA; I finished in 42:44, which was 97th in a field of 1633.  I beat or
tied 1537 runners out of 1633, so my percentile rank in the field is
94\%.  \index{James Joyce Ramble} \index{race time}

More generally, given position and field size, we can compute
percentile rank:
\index{field size}

\begin{verbatim}
def PositionToPercentile(position, field_size):
    beat = field_size - position + 1
    percentile = 100.0 * beat / field_size
    return percentile
\end{verbatim}

In my age group, denoted M4049 for ``male between 40 and 49 years of
age'', I came in 26th out of 256.  So my percentile rank in my age
group was 90\%.
\index{age group}

If I am still running in 10 years (and I hope I am), I will be in
the M5059 division.  Assuming that my percentile rank in my division
is the same, how much slower should I expect to be?

I can answer that question by converting my percentile rank in M4049
to a position in M5059.  Here's the code:

\begin{verbatim}
def PercentileToPosition(percentile, field_size):
    beat = percentile * field_size / 100.0
    position = field_size - beat + 1
    return position
\end{verbatim}

There were 171 people in M5059, so I would have to come in between
17th and 18th place to have the same percentile rank.  The finishing
time of the 17th runner in M5059 was 46:05, so that's the time I will
have to beat to maintain my percentile rank.


\section{Exercises}

For the following exercises, you can start with \verb"chap04ex.ipynb".
My solution is in \verb"chap04soln.ipynb".

\begin{exercise}
How much did you weigh at birth?  If you don't know, call your mother
or someone else who knows.  Using the NSFG data (all live births),
compute the distribution of birth weights and use it to find your
percentile rank.  If you were a first baby, find your percentile rank
in the distribution for first babies.  Otherwise use the distribution
for others.  If you are in the 90th percentile or higher, call your
mother back and apologize.
\index{birth weight}
\index{weight!birth}

\end{exercise}

\begin{exercise}
The numbers generated by {\tt random.random} are supposed to be
uniform between 0 and 1; that is, every value in the range
should have the same probability.

Generate 1000 numbers from {\tt random.random} and plot their
PMF and CDF.  Is the distribution uniform?
\index{uniform distribution}
\index{distribution!uniform}
\index{random number}

\end{exercise}


\section{Glossary}

\begin{itemize}

\item percentile rank: The percentage of values in a distribution that are
less than or equal to a given value.
\index{percentile rank}

\item percentile: The value associated with a given percentile rank.
\index{percentile}

\item cumulative distribution function (CDF): A function that maps
  from values to their cumulative probabilities.  $\CDF(x)$ is the
  fraction of the sample less than or equal to $x$.  \index{CDF}
\index{cumulative probability}

\item inverse CDF: A function that maps from a cumulative probability,
  $p$, to the corresponding value.
\index{inverse CDF}
\index{CDF, inverse}

\item median: The 50th percentile, often used as a measure of central
  tendency.  \index{median}

\item interquartile range: The difference between
the 75th and 25th percentiles, used as a measure of spread.
\index{interquartile range}

\item quantile: A sequence of values that correspond to equally spaced
percentile ranks; for example, the quartiles of a distribution are
the 25th, 50th and 75th percentiles.
\index{quantile}

\item replacement: A property of a sampling process. ``With replacement''
means that the same value can be chosen more than once; ``without
replacement'' means that once a value is chosen, it is removed from
the population.
\index{replacement}

\end{itemize}

