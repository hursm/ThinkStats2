\chapter{분포 모형화 (Modeling distributions)}
\label{modeling}

지금까지 사용한 분포는 {\bf 경험적 분포 (empirical distributions)}라고 부른다.
이유는 필연적으로 유한 표본인 경험적 관측치에 기반하고 있기 때문이다.

\index{해석 분포 (analytic distribution)}
\index{분포 (distribution)!해석 (analytic)}
\index{경험적 분포 (empirical distribution)}
\index{분포 (distribution)!경험 (empirical)}

수학 함수인 CDF로 특징 지어지는 {\bf 해석 분포 (analytic distribution)}가 대안이 된다.
해석 분포가 경험적 분포를 모형화하는데 사용될 수 있다.
이러한 맥락에서 {\bf 모형(model)}은 불필요한 부분을 덜어낸 단순화가 된다.
이번 장에서 자주 사용되는 분포를 제시하고 이를 사용하여 다양한 출처를 가진 
데이터를 모형화한다.

\index{모형 (model)}

이번 장에서 사용되는 코드는 {\tt analytic.py}에 있다.
코드를 다운로드하고 작업하는 것에 대한 정보는 ~\ref{code}을 참조한다.


\section{지수분포 (exponential distribution)}
\label{exponential}
\index{지수분포 (exponential distribution)}
\index{분포 (distribution)!지수 (exponential)}

\begin{figure}
% analytic.py
%\centerline{\includegraphics[height=2.5in]{figs/analytic_expo_cdf.pdf}}
\caption{CDFs of exponential distributions with various parameters.}
\label{analytic_expo_cdf}
\end{figure}

{\bf 지수 분포 (exponential distribution)}로 시작하는데 이유는 상대적으로 단순하기 때문이다.
지수분포 CDF는 다음과 같다.
%
\[ \CDF(x) = 1 - e^{-\lambda x} \]
%

모수 $\lambda$가 분포 형상(shape)을 결정한다. 
그림~\ref{analytic_expo_cdf}에서 $\lambda = $ 0.5, 1, 2 값을 가진
CDF가 대략 모양이 어떤지 볼 수 있다.
\index{모수 (parameter)}

현실 세계에서 일련의 사건을 보고, 사건 간에 시간({\bf 도착간격 시간, interarrival times})을 측정할 때 지수분포가 등장한다.
만약 사건이 언제든지 균등하게 발생할 것 같다면 도착간격 시간 분포는 지수분포같은 경향이 있다.
\index{도착간격 시간 (interarrival time)}

일례로, 출생간 발생시간을 살펴보자. 1997년 12월 18일 호주 브리즈번
\footnote{예제에 나오는 자료와 정보는 저널 논문에 기반한다. Dunn, ``A Simple Dataset for Demonstrating Common Distributions,'' Journal of Statistics Education v.7, n.3 (1999)}
에서 44명 신생아가 출생했다. 모든 44명 신생아 출생 시간이 지역신문에 출간되었다;
전체 데이터셋은 {\tt ThinkStats2} 저장소 {\tt babyboom.dat} 파일에 담겨있다.
\index{출생 시간 (birth time)}
\index{호주 (Australia)} 
\index{브리즈번 (Brisbane)}

\begin{verbatim}
    df = ReadBabyBoom()
    diffs = df.minutes.diff()
    cdf = thinkstats2.Cdf(diffs, label='actual')

    thinkplot.Cdf(cdf)
    thinkplot.Show(xlabel='minutes', ylabel='CDF')
\end{verbatim}

{\tt ReadBabyBoom} 함수가 데이터 파일을 읽어들이고 {\tt time}, {\tt sex}, \verb"weight_g", {\tt minutes}
칼럼으로 구성된 데이터프레임을 반환한다.
여기서 {\tt minutes}가 자정 이후 출생시간을 분으로 변환한 시간정보를 담고 있다.
\index{데이터프레임 (DataFrame)}
\index{thinkplot}

\begin{figure}
% analytic.py
%\centerline{\includegraphics[height=2.5in]{figs/analytic_interarrivals.pdf}}
\caption{CDF of interarrival times (left) and CCDF on a log-y scale (right).}
\label{analytic_interarrival_cdf}
\end{figure}

%\begin{figure}
% analytic.py
%%\centerline{\includegraphics[height=2.5in]{figs/analytic_interarrivals_logy.pdf}}
%\caption{CCDF of interarrival times.}
%\label{analytic_interarrival_ccdf}
%\end{figure}

{\tt diffs}는 연속되는 출생시간 사이 차이가 되고 
{\tt cdf}는 출생간격 시간 분포가 된다.
그림~\ref{analytic_interarrival_cdf} (왼편)이 CDF를 나타낸다.
전형적인 지수분포 형상을 지닌 처럼 보이지만, 어떻게 분간할 수 있을까?

한 방법은 {\bf 보완 CDF (complementary CDF)}를 플롯으로 그리는 것이다.
보완 CDF는 log-y 척도로 $1 - \CDF(x)$이다.
지수분포 데이터에 대해서는 결과가 직선이다. 왜 그런지 살펴보자.

\index{보완 CDF (complementary CDF)} 
\index{CDF!보완 (complementary)} 
\index{CCDF}

독자가 생각하기에 지수분포를 따르는 데이터셋을 보완 CDF(CCDF) 플롯으로 그리면, 
다음과 같은 함수가 나올 것으로 기대한다.
%
\[ y \approx e^{-\lambda x} \]
%
양변에 로그를 취하면 다음과 같다.
%
\[ \log y \approx -\lambda x\]
%
그래서, log-y 척도로 CCDF는 기울기 $-\lambda$인 직선이 된다.
다음에 플롯을 생성하는 방법이 있다.
\index{로그 척도 (logarithmic scale)}
\index{보완 CDF (complementary CDF)}
\index{CDF!보완 (complementary)}
\index{CCDF}

\begin{verbatim}
    thinkplot.Cdf(cdf, complement=True)
    thinkplot.Show(xlabel='minutes',
                   ylabel='CCDF',
                   yscale='log')
\end{verbatim}

{\tt complement=True} 인자가 있어서, {\tt thinkplot.Cdf}이 플롯을 그리기 전에
보완 CDF를 계산한다. 그리고 {\tt yscale='log'}를 통해서  
{\tt thinkplot.Show}가 로그 척도로 {\tt y}축을 고정한다.
\index{thinkplot}
\index{Cdf}

그림~\ref{analytic_interarrival_cdf} (오른편)에 결과가 있다.
정확하게 직선이 아니다. 이 데이터에 대해서 완벽한 모델로 지수 분포가 아니라는 것이 표시된다.
기본 가정---출생이 아무 때고 균등하게 발생---이 정확하게 사실이 아닐 것이다.
그럼에도 불구하고 지수분포로 이 데이터셋을 모형화하는 것이 합리적일 것이다.
이와 같은 단순화로 단 하나의 모수로 분포를 요약할 수 있다.
\index{모형 (model)}

모수 $\lambda$가 율(rate)로 해석될 수 있다; 즉, 평균적으로 단위 시간에 
발생하는 사건 수. 예제에서 44명의 신생아가 24시간내에 태어난다.
그래서 율값이 분당 $\lambda = 0.0306$이 된다.
지수분포 평균은 $1/\lambda$ 으로 신생아 간에 출생 평균 시간은 32.7분이 된다.

\section{정규 분포 (normal distribution)}
\label{normal}

가우스 분포(Gaussian distribution)라고도 불리는 
{\bf 정규 분포 (normal distribution)}가 흔히 사용되는데 이유는 많은 현상을 기술하고 
최소한 근사적으로도 기술할 수 있기 때문이다.
\ref{CLT} 절에서 다루게 되는데 이와 같은 보편성에는 이유가 있다.
\index{CDF}
\index{모수 (parameter)}
\index{평균 (mean)}
\index{표준편차 (standard deviation)}
\index{정규 분포 (normal distribution)}
\index{분포 (distribution)!정규 (normal)}
\index{가우스 분포 (Gaussian distribution)}
\index{분포 (distribution)!가우스 (Gaussian)}

%
\[ \CDF(z) = \frac{1}{\sqrt{2 \pi}} \int_{-\infty}^z e^{-t^2/2} dt \]
%

\begin{figure}
% analytic.py
%\centerline{\includegraphics[height=2.5in]{figs/analytic_gaussian_cdf.pdf}}
\caption{CDF of normal distributions with a range of parameters.}
\label{analytic_gaussian_cdf}
\end{figure}

정규 분포는 모수 두개로 특성화된다: 평균 $\mu$, 표준편차 $\sigma$.
모수 $\mu=0$과 $\sigma=1$을 갖는 정규분포를 {\bf 표준 정규 분포 (standard normal
 distribution)}라고 한다.
정규분포 CDF는 닫힌 형식 해법(closed form solution)을 갖지 않는 적분으로 정의된다.
하지만, 효율적으로 계산하는 알고리즘이 있다.
알고리즘 중 하나가 SciPy을 통해 제공된다: {\tt scipy.stats.norm}이
정규분포를 표현하는 객체다. 표준 정규분포 CDF를 계산하는 {\tt cdf} 메쏘드를 제공한다.

\index{SciPy}
\index{닫힌 형식 (closed form)}

\begin{verbatim}
>>> import scipy.stats
>>> scipy.stats.norm.cdf(0)
0.5
\end{verbatim}

결과값은 맞다: 표준 정규분포 중위수는 0 (평균과 같다)이고, 값의 절반이 중위수 아래 위치한다.
그래서 $\CDF(0)$은 0.5 이다.

{\tt norm.cdf}은 옵션 모수를 받는다: {\tt loc}가
평균을 특정하고, {\tt scale}는 표준편차를 특정한다.


{\tt thinkstats2}는 상기 함수를 좀더 사용하기 쉽게 한다.
{\tt EvalNormalCdf} 메쏘드는 {\tt mu}과 {\tt sigma}을 인자로 받아 
{\tt x}에 CDF를 계산한다.

\index{정규 분포 (normal distribution)}

\begin{verbatim}
def EvalNormalCdf(x, mu=0, sigma=1):
    return scipy.stats.norm.cdf(x, loc=mu, scale=sigma)
\end{verbatim}

그림~\ref{analytic_gaussian_cdf}에 정규분포에 다양한 모수를 넣어 그린 CDF가 있다.
곡선의 S자(sigmoid) 형상이 정규분포를 식별할 수 있게 하는 특성이다.

앞장에서 NSFG 출생 체중 분포를 살펴봤다.
모든 정상 출산 체중에 대한 경험적 CDF와 동일한 평균과 분산으로 정규분포 CDF를 중첩하여 
플롯으로 그린 것이 그림~\ref{analytic_birthwgt_model}이다.

\index{가족 성장 국가 조사 (National Survey of Family Growth)}
\index{NSFG}
\index{출생 체중 (birth weight)}
\index{체중 (weight)!출생 (birth)}

\begin{figure}
% analytic.py
%\centerline{\includegraphics[height=2.5in]{figs/analytic_birthwgt_model.pdf}}
\caption{CDF of birth weights with a normal model.}
\label{analytic_birthwgt_model}
\end{figure}

정규분포가 이 데이터셋에 대해서 좋은 모형이 된다.
그래서 만약 모수 $\mu = 7.28$과 $\sigma = 1.24$으로 
분포를 요약한다면, 결과 오차(모형과 데이터 간 차이)가 작다.

\index{모형 (model)}
\index{백분위수 (percentile)}

백분위수 10번째 아래에서 데이터와 모형 사이에 불일치가 있다.
정규분포에서 예측되는 것보다 더 많이 체중이 적은 아이가 있다.
만약 조산아에 특별히 관심이 있다면, 분포에서 이 부분을 잘 적합하는 것이 중요하다.
그래서 정규분포를 사용하는 것이 적절하지 않을 수도 있다.


\section{정규확률그림 (Normal probability plot)}

지수분포와 몇가지 분포에서 대해서 해석분포(analytic distribution)가 특정 데이터셋에 대해서
적합한 모형인가를 테스트하는데 사용할 수 있는 간단한 변환이 있다.

\index{지수분포 (exponential distribution)}
\index{분포 (distribution)!지수 (exponential)}
\index{모형 (model)}

정규분포에 대해서 그러한 변환은 없다. 하지만, 
{\bf 정규확률그림 (normal probability plot)}으로 불리는 대안이 있다.
정규확률그림을 생성하는 방식이 두개 있다.: 어려운 방식과 쉬운 방식.
어려운 방식에 관심이 있다면 \url{https://en.wikipedia.org/wiki/Normal_probability_plot}에서 
자세한 정보를 얻을 수 있다.
다음에 쉬운 방식이 있다.
\index{정규확률그림 (normal probability plot)}
\index{그림 (plot)! 정규확률 (normal probability)}
\index{정규분포 (normal distribution)}
\index{분포 (distribution)!정규 (normal)}
\index{가우스 분포 (Gaussian distribution)}
\index{분포 (distribution)!가우스 (Gaussian)}

\begin{enumerate}

\item 표본에 있는 값을 정렬한다.

\item 표준정규분포($\mu=0$, $\sigma=1$)에서 표본과 동일한 크기를 갖는 난수을 생성하고
정렬한다.
\index{난수 (random number)}

\item 표본에서 나온 정렬된 값과 난수를 플롯으로 그린다.

\end{enumerate}

만약 표본 분포가 근사적으로 정규분포라면, 결과는 
절편 {\tt mu}, 기울기 {\tt sigma}를 갖는 직선이다.
{\tt thinkstats2}에 {\tt NormalProbability}이 있다.
표본을 인자로 받아서 넘파이(NumPy) 배열 두개를 반환한다.
\index{넘파이 (NumPy)}

\begin{verbatim}
xs, ys = thinkstats2.NormalProbability(sample)
\end{verbatim}

\begin{figure}
% analytic.py
%\centerline{\includegraphics[height=2.5in]{figs/analytic_normal_prob_example.pdf}}
\caption{Normal probability plot for random samples from normal distributions.}
\label{analytic_normal_prob_example}
\end{figure}

{\tt ys}는 {\tt sample}에서 정렬된 값이 담겨있다; 
{\tt xs}에는 표준정규분포에서 생성된 난수가 담겨있다.

{\tt NormalProbability}을 테스트하기 위해서 다양한 모수를 가진 
정규분포에서 모조 샘플을 생성했다.
그림~\ref{analytic_normal_prob_example}에 결과가 있다.
선들이 근사적으로 직선으로, 평균에 있는 값보다 벗어난 값을 꼬리에 갖는다.

이제 실제 데이터에 적합을 시도해 보자.
앞절로부터 출생 체중 데이터에 대해 정규확률그림을 생성하는 코드가 다음에 있다.
모형을 표현하는 회색선과 실제 데이터를 표현하는 파란선을 플롯으로 그린다.

\index{출생 체중 (birth weight)}
\index{체중 (weight)!출생 (birth)}

\begin{verbatim}
def MakeNormalPlot(weights):
    mean = weights.mean()
    std = weights.std()

    xs = [-4, 4]
    fxs, fys = thinkstats2.FitLine(xs, inter=mean, slope=std)
    thinkplot.Plot(fxs, fys, color='gray', label='model')

    xs, ys = thinkstats2.NormalProbability(weights)
    thinkplot.Plot(xs, ys, label='birth weights')
\end{verbatim}

{\tt weights}는 출생 체중 판다스 시리즈다; {\tt mean}과 {\tt std}은
각각 평균과 표준편차다.
\index{판다스 (pandas)}
\index{시리즈 (Series)}
\index{thinkplot}
\index{표준편차 (standard deviation)}

{\tt FitLine}이 시퀀스 {\tt xs}, 절편, 기울기를 인자로 받는다;
반환하는 {\tt xs}와 {\tt ys}는 {\tt xs} 값에서 계산되어 인자로 받은 모수를 가진 직선이다.

{\tt NormalProbability}은 {\tt xs}와 {\tt ys}를 반환하는데 
표준정규분포에서 나온 값과 {\tt weights}에서 나온 값을 담고 있다.
만약 체중 분포가 정규분포를 따른다면, 데이터도 모델과 매칭되어야 한다.

\index{모형 (model)}

\begin{figure}
% analytic.py
%\centerline{\includegraphics[height=2.5in]{figs/analytic_birthwgt_normal.pdf}}
\caption{Normal probability plot of birth weights.}
\label{analytic_birthwgt_normal}
\end{figure}

그림~\ref{analytic_birthwgt_normal}이 전체 정상 출생과 더불어 만삭(임신기간이 36주 이상)에 대한 결과를 보여준다.
두 곡선 모두 평균 근처에서 모형과 매칭되고 꼬리에서 차이가 난다.
가장 무거운 아이가 모형이 예측한 것보다 더 무겁고, 가장 가벼운 아이는 더 가볍다.

\index{임신 기간}

단지 만삭 아이만 선택해서, 가장 가벼운 몇몇 아이를 제거하면 분포 아래쪽 꼬리에 있는 불일치를 줄일 수 있다.

정규분포 모형이 평균에서부터 몇 표준편차 내에서 분포를 잘 기술하지만, 꼬리 부근에서는 아니라고 플롯 그래프를 통해서 알 수 있다.
실제 목적에 얼마나 부합되는지는 목적에 달려있다.
\index{모형 (model)}
\index{출생 체중 (birth weight)}
\index{체중 (weight)!출생 (birth)}
\index{표준편차 (standard deviation)}


\section{로그 정규분포 (lognormal distribution)}
\label{brfss}
\label{lognormal}

만약 값을 로그 취한 집합이 정규분포라면, 이 값은 {\bf 로그 정규분포 (lognormal distribution)}다. 로그 정규분포 CDF는 $x$를 $\log x$로 치환한 정규분포 CDF와 동일하다.
%
\[ CDF_{lognormal}(x) = CDF_{normal}(\log x)\]
%
로그 정규분포 모수는 일반적으로 $\mu$와 $\sigma$로 표기한다.
하지만, 기억할 것은 모수가 평균과 표준편차는 {\em 아니다};
로그 정규분포 평균은 $\exp(\mu +\sigma^2/2)$이고, 표준편차는 조금 복잡한다.(\url{http://wikipedia.org/wiki/Log-normal_distribution}를 참조한다)
\index{모수 (parameter)} 
\index{체중 (weight)!성인 (adult)} 
\index{성인 체중 (adult weight)}
\index{로그 정규분포 (lognormal distribution)}
\index{분포 (distribution)!로그 정규 (lognormal)}
\index{CDF}

\begin{figure}
% brfss.py
%\centerline{
%\includegraphics[height=2.5in]{figs/brfss_weight.pdf}}
%\caption{CDF of adult weights on a linear scale (left) and
%log scale (right).}
%\label{brfss_weight}
\end{figure}

만약 표본이 근사적으로 로그 정규분포이고 log-x 척도로 CDF 플롯을 그린다면, 정규분포 특성의 형상을 갖는다.
표본이 로그 정규분포 모형과 얼마나 잘 적합하는지 테스트하기 위해서, 
표본값에 로그를 취해서 정규확률그림을 생성할 수 있다.

\index{정규확률그림 (normal probability plot)}
\index{모형 (model)}

예제로 성인 체중 분포를 살펴보는데 근사적으로 로그 정규분포다.\footnote{\url{http://mathworld.wolfram.com/LogNormalDistribution.html} 사이트에서 주석(인용없이)으로 이 가능성에 대해서 제보를 받았다.
나중에 로그 변환과 원인을 제안하는 논문을 발견했다: Penman and Johnson, ``The Changing Shape of the Body Mass Index Distribution Curve in the Population,'' Preventing Chronic Disease, 2006 July; 3(3): A74.  Online at \url{http://www.ncbi.nlm.nih.gov/pmc/articles/PMC1636707}.}


만성 질환 예방 및 건강 증진을 위한 국립 센터 (National Center for Chronic Disease Prevention and Health Promotion)에서는 
행동 위험 요인 감시 시스템(Behavioral Risk Factor Surveillance System, BRFSS)\footnote{질병통제 예방센터(Centers for Disease Control and Prevention, CDC). 행동 위험 요인 감시 시스템 조사 자료(Behavioral Risk Factor Surveillance System Survey Data). Atlanta, Georgia: 미국 보건 복지부 (U.S. Department of Health and Human Services), 질병통제 예방센터 (Centers for Disease Control and Prevention), 2008.}의 일부문으로 매년 조사를 실시한다.
2008년 414,509 응답자를 대상으로 인터뷰했고 인구통계, 건강, 건강 위험에 관해서 설문했다. 수집한 데이터 중에 398,484 응답자 킬로그램으로 표시된 체중정보가 있다.
\index{행동 위험 요인 감시 시스템 (Behavioral Risk Factor Surveillance System)}
\index{BRFSS}

이 책을 위한 저장소에 BRFSS에 관한 데이터를 담고 있는 고정폭 아스키(ASCII)파일, {\tt CDBRFS08.ASC.gz}와 더불어 파일을 읽고 데이터를 분석하는 {\tt brfss.py}파일도 함께 있다.

\begin{figure}
% brfss.py
%\centerline{
%\includegraphics[height=2.5in]{figs/brfss_weight_normal.pdf}}
\caption{Normal probability plots for adult weight on a linear scale
  (left) and log scale (right).}
\label{brfss_weight_normal}
\end{figure}

그림~\ref{brfss_weight} (왼편)이 정규분포 모형에 선형 척도로 성인 체중 분포를 나타낸다.
그림~\ref{brfss_weight} (오른편)이 로그 정규분포 모형으로 로그 척도로 동일한 분포를 나타낸다.
로그 정규 모형이 더 나은 적합이지만 데이터를 이와 같이 표현하는 것이 차이점을 특별히 인상적으로 만들지는 못한다.
\index{응답자 (respondent)} 
\index{모형 (model)}

그림~\ref{brfss_weight_normal}이 성인 체중 $w$에 대한 정규확률그림과 
로그 변환한 체중 $\log_{10} w$에 대한 정규확률그림을 보여준다.
이제 데이터가 정규분포 모형에서 상당히 벗아난 것이 명확하다.
다른 한편으로 로그 정규모형은 데이터에 대한 좋은 매칭을 보여준다.

\index{정규분포 (normal distribution)} 
\index{분포 (distribution)!정규 (normal0}
\index{가우스 분포 (Gaussian distribution)} 
\index{분포 (distribution)!가우스 (Gaussian)}
\index{로그 정규분포 (lognormal distribution)} 
\index{분포 (distribution)!로그 정규 (lognormal)}
\index{표준편차 (standard deviation)} 
\index{성인 체중 (adult weight)} 
\index{체중 (weight)!성인 (adult)}
\index{모형 (model)} 
\index{정규확률그림 (normal probability plot)}


\section{파레토 분포 (Pareto distribution)}
\index{파레토 분포 (Pareto distribution)}
\index{분포 (distribution)!파레토 (Pareto)}
\index{파레토, 빌프레도 (Pareto, Vilfredo)}

The {\bf Pareto distribution} is named after the economist Vilfredo Pareto,
who used it to describe the distribution of wealth (see
\url{http://wikipedia.org/wiki/Pareto_distribution}).  Since then, it
has been used to describe phenomena in the natural and social sciences
including sizes of cities and towns, sand particles and meteorites,
forest fires and earthquakes.  \index{CDF}

The CDF of the Pareto distribution is:
%
\[ CDF(x) = 1 - \left( \frac{x}{x_m} \right) ^{-\alpha} \]
%
The parameters $x_{m}$ and $\alpha$ determine the location and shape
of the distribution. $x_{m}$ is the minimum possible value.
Figure~\ref{analytic_pareto_cdf} shows CDFs of Pareto
distributions with $x_{m} = 0.5$ and different values
of $\alpha$.
\index{parameter}

\begin{figure}
% analytic.py
%\centerline{\includegraphics[height=2.5in]{figs/analytic_pareto_cdf.pdf}}
\caption{CDFs of Pareto distributions with different parameters.}
\label{analytic_pareto_cdf}
\end{figure}

There is a simple visual test that indicates whether an empirical
distribution fits a Pareto distribution: on a log-log scale, the CCDF
looks like a straight line.  Let's see why that works.

If you plot the CCDF of a sample from a Pareto distribution on a
linear scale, you expect to see a function like:
%
\[ y \approx \left( \frac{x}{x_m} \right) ^{-\alpha} \]
%
Taking the log of both sides yields:
%
\[ \log y \approx -\alpha (\log x - \log x_{m})\]
%
So if you plot $\log y$ versus $\log x$, it should look like a straight
line with slope $-\alpha$ and intercept
$\alpha \log x_{m}$.

As an example, let's look at the sizes of cities and towns.
The U.S.~Census Bureau publishes the
population of every incorporated city and town in the United States.
\index{Pareto distribution} \index{distribution!Pareto}
\index{U.S.~Census Bureau} \index{population} \index{city size}

\begin{figure}
% populations.py
%\centerline{\includegraphics[height=2.5in]{figs/populations_pareto.pdf}}
\caption{CCDFs of city and town populations, on a log-log scale.}
\label{populations_pareto}
\end{figure}

I downloaded their data from
\url{http://www.census.gov/popest/data/cities/totals/2012/SUB-EST2012-3.html};
it is in the repository for this book in a file named
\verb"PEP_2012_PEPANNRES_with_ann.csv".  The repository also
contains {\tt populations.py}, which reads the file and plots
the distribution of populations.

Figure~\ref{populations_pareto} shows the CCDF of populations on a
log-log scale.  The largest 1\% of cities and towns, below $10^{-2}$,
fall along a straight line.  So we could
conclude, as some researchers have, that the tail of this distribution
fits a Pareto model.
\index{model}

On the other hand, a lognormal distribution also models the data well.
Figure~\ref{populations_normal} shows the CDF of populations and a
lognormal model (left), and a normal probability plot (right).  Both
plots show good agreement between the data and the model.
\index{normal probability plot}

Neither model is perfect.
The Pareto model only applies to the largest 1\% of cities, but it
is a better fit for that part of the distribution.  The lognormal
model is a better fit for the other 99\%.
Which model is appropriate depends on which part of the distribution
is relevant.

\begin{figure}
% populations.py
%\centerline{\includegraphics[height=2.5in]{figs/populations_normal.pdf}}
\caption{CDF of city and town populations on a log-x scale (left), and
normal probability plot of log-transformed populations (right).}
\label{populations_normal}
\end{figure}


\section{Generating random numbers}
\index{exponential distribution}
\index{distribution!exponential}
\index{random number}
\index{CDF}
\index{inverse CDF algorithm}
\index{uniform distribution}
\index{distribution!uniform}

Analytic CDFs can be used to generate random numbers with a given
distribution function, $p = \CDF(x)$.  If there is an efficient way to
compute the inverse CDF, we can generate random values
with the appropriate distribution by choosing $p$ from a uniform
distribution between 0 and 1, then choosing
$x = ICDF(p)$.
\index{inverse CDF}
\index{CDF, inverse}

For example, the CDF of the exponential distribution is
%
\[ p = 1 - e^{-\lambda x} \]
%
Solving for $x$ yields:
%
\[ x = -\log (1 - p) / \lambda \]
%
So in Python we can write
%
\begin{verbatim}
def expovariate(lam):
    p = random.random()
    x = -math.log(1-p) / lam
    return x
\end{verbatim}

{\tt expovariate} takes {\tt lam} and returns a random value chosen
from the exponential distribution with parameter {\tt lam}.

Two notes about this implementation:
I called the parameter \verb"lam" because \verb"lambda" is a Python
keyword.  Also, since $\log 0$ is undefined, we have to
be a little careful.  The implementation of {\tt random.random}
can return 0 but not 1, so $1 - p$ can be 1 but not 0, so
{\tt log(1-p)} is always defined.  \index{random module}


\section{Why model?}
\index{model}

At the beginning of this chapter, I said that many real world phenomena
can be modeled with analytic distributions.  ``So,'' you might ask,
``what?''  \index{abstraction}

Like all models, analytic distributions are abstractions, which
means they leave out details that are considered irrelevant.
For example, an observed distribution might have measurement errors
or quirks that are specific to the sample; analytic models smooth
out these idiosyncrasies.
\index{smoothing}

Analytic models are also a form of data compression.  When a model
fits a dataset well, a small set of parameters can summarize a
large amount of data.
\index{parameter}
\index{compression}

It is sometimes surprising when data from a natural phenomenon fit an
analytic distribution, but these observations can provide insight
into physical systems.  Sometimes we can explain why an observed
distribution has a particular form.  For example, Pareto distributions
are often the result of generative processes with positive feedback
(so-called preferential attachment processes: see
\url{http://wikipedia.org/wiki/Preferential_attachment}.).
\index{preferential attachment}
\index{generative process}
\index{Pareto distribution}
\index{distribution!Pareto}
\index{analysis}

Also, analytic distributions lend themselves to mathematical
analysis, as we will see in Chapter~\ref{analysis}.

But it is important to remember that all models are imperfect.
Data from the real world never fit an analytic distribution perfectly.
People sometimes talk as if data are generated by models; for example,
they might say that the distribution of human heights is normal,
or the distribution of income is lognormal.  Taken literally, these
claims cannot be true; there are always differences between the
real world and mathematical models.

Models are useful if they capture the relevant aspects of the
real world and leave out unneeded details.  But what is ``relevant''
or ``unneeded'' depends on what you are planning to use the model
for.


\section{Exercises}

For the following exercises, you can start with \verb"chap05ex.ipynb".
My solution is in \verb"chap05soln.ipynb".

\begin{exercise}
In the BRFSS (see Section~\ref{lognormal}), the distribution of
heights is roughly normal with parameters $\mu = 178$ cm and
$\sigma = 7.7$ cm for men, and $\mu = 163$ cm and $\sigma = 7.3$ cm for
women.
\index{normal distribution}
\index{distribution!normal}
\index{Gaussian distribution}
\index{distribution!Gaussian}
\index{height}
\index{Blue Man Group}
\index{Group, Blue Man}

In order to join Blue Man Group, you have to be male between 5'10''
and 6'1'' (see \url{http://bluemancasting.com}).  What percentage of
the U.S. male population is in this range?  Hint: use {\tt
  scipy.stats.norm.cdf}.
\index{SciPy}

\end{exercise}


\begin{exercise}
To get a feel for the Pareto distribution, let's see how different
the world
would be if the distribution of human height were Pareto.
With the parameters $x_{m} = 1$ m and $\alpha = 1.7$, we
get a distribution with a reasonable minimum, 1 m,
and median, 1.5 m.
\index{height}
\index{Pareto distribution}
\index{distribution!Pareto}

Plot this distribution.  What is the mean human height in Pareto
world?  What fraction of the population is shorter than the mean?  If
there are 7 billion people in Pareto world, how many do we expect to
be taller than 1 km?  How tall do we expect the tallest person to be?
\index{Pareto World}

\end{exercise}


\begin{exercise}
\label{weibull}

The Weibull distribution is a generalization of the exponential
distribution that comes up in failure analysis
(see \url{http://wikipedia.org/wiki/Weibull_distribution}).  Its CDF is
%
\[ CDF(x) = 1 - e^{-(x / \lambda)^k} \]
%
Can you find a transformation that makes a Weibull distribution look
like a straight line?  What do the slope and intercept of the
line indicate?
\index{Weibull distribution}
\index{distribution!Weibull}
\index{exponential distribution}
\index{distribution!exponential}
\index{random module}

Use {\tt random.weibullvariate} to generate a sample from a
Weibull distribution and use it to test your transformation.

\end{exercise}


\begin{exercise}
For small values of $n$, we don't expect an empirical distribution
to fit an analytic distribution exactly.  One way to evaluate
the quality of fit is to generate a sample from an analytic
distribution and see how well it matches the data.
\index{empirical distribution}
\index{distribution!empirical}
\index{random module}

For example, in Section~\ref{exponential} we plotted the distribution
of time between births and saw that it is approximately exponential.
But the distribution is based on only 44 data points.  To see whether
the data might have come from an exponential distribution, generate 44
values from an exponential distribution with the same mean as the
data, about 33 minutes between births.

Plot the distribution of the random values and compare it to the
actual distribution.  You can use {\tt random.expovariate} 
to generate the values.

\end{exercise}

\begin{exercise}
In the repository for this book, you'll find a set of data files
called {\tt mystery0.dat}, {\tt mystery1.dat}, and so on.  Each
contains a sequence of random numbers generated from an analytic
distribution.
\index{random number}

You will also find \verb"test_models.py", a script that reads
data from a file and plots the CDF under a variety of transforms.
You can run it like this:

\begin{verbatim}
$ python test_models.py mystery0.dat
\end{verbatim}

Based on these plots, you should be able to infer what kind of
distribution generated each file.  If you are stumped, you can
look in {\tt mystery.py}, which contains the code that generated
the files.

\end{exercise}


\begin{exercise}
\label{income}

The distributions of wealth and income are sometimes modeled using
lognormal and Pareto distributions.  To see which is better, let's
look at some data.
\index{Pareto distribution}
\index{distribution!Pareto}
\index{lognormal distribution}
\index{distribution!lognormal}

The Current Population Survey (CPS) is a joint effort of the Bureau
of Labor Statistics and the Census Bureau to study income and related
variables.  Data collected in 2013 is available from
\url{http://www.census.gov/hhes/www/cpstables/032013/hhinc/toc.htm}.
I downloaded {\tt hinc06.xls}, which is an Excel spreadsheet with
information about household income, and converted it to {\tt hinc06.csv},
a CSV file you will find in the repository for this book.  You
will also find {\tt hinc.py}, which reads this file.

Extract the distribution of incomes from this dataset.  Are any of the
analytic distributions in this chapter a good model of the data?  A
solution to this exercise is in \url{hinc_soln.py}.
\index{model}

\end{exercise}




\section{Glossary}

\begin{itemize}

\item empirical distribution: The distribution of values in a sample.
  \index{empirical distribution} \index{distribution!empirical}

\item analytic distribution: A distribution whose CDF is an analytic
function.
\index{analytic distribution}
\index{distribution!analytic}

\item model: A useful simplification.  Analytic distributions are
often good models of more complex empirical distributions.
\index{model}

\item interarrival time: The elapsed time between two events.
\index{interarrival time}

\item complementary CDF: A function that maps from a value, $x$,
to the fraction of values that exceed $x$, which is $1 - \CDF(x)$.
\index{complementary CDF} \index{CDF!complementary} \index{CCDF}

\item standard normal distribution: The normal distribution with
mean 0 and standard deviation 1.
\index{standard normal distribution}

\item normal probability plot: A plot of the values in a sample versus
random values from a standard normal distribution.
\index{normal probability plot}
\index{plot!normal probability}

\end{itemize}

