
\chapter{선형최소제곱 (Linear least squares)}
\label{linear}

이번 장에서 사용되는 코드는 {\tt linear.py}에 있다.
코드를 다운로드하고 작업하는 것에 대한 정보는 ~\ref{code}을 참조한다.

\section{최소제곱 적합 (Least squares fit)}

상관계수는 관계 부호와 강도를 측정하지만 기울기는 측정하지 않는다.
기울기를 측정하는 방법이 몇가지 있다; 가장 흔한 방법이 {\bf 선형최소제곱 적합 (linear least squares fit)}이다. ``선형 적합(linear fit)''은 변수 사이 관계를 모형화하는 선(line)이다.
``최소제곱 (least squares)'' 적합은 선과 데이터 사이 평균제곱오차(mean
squared error, MSE)를 최소화하는 것이다.
\index{최소제곱 적합 (least squares fit)}
\index{선형최소제곱 (linear least squares)}
\index{모형 (model)}

하나 시퀀스 {\tt ys}가 있는데, 또 다른 시퀀스 {\tt xs} 함수로 표현하고자 한다고 가정하자.
만약 {\tt xs}, {\tt ys}와 절편 {\tt inter}, 기울기 {\tt slope} 사이에 선형 관계가 있다면,
각 {\tt y[i]} 가 {\tt inter + slope * x[i]}이 될 것으로 예상된다.  
\index{잔차 (residuals)}

하지만, 상관이 완벽하지 않다면, 예측은 단지 근사(approximation)가 된다.
선에서 수직 편차(vertical deviation), 즉 {\bf 잔차(residual)}는 다음과 같다. 
\index{편차 (deviation)}

\begin{verbatim}
res = ys - (inter + slope * xs)
\end{verbatim}

잔차는 측정 오차 같은 확률 요소(random factor), 혹은 알지 못하는 비임의 요소(non-random factor) 때문일지 모른다. 예를 들어, 만약 체중을 신장 함수로 예측한다면, 미지 요소는 식습관, 운동, 신체 유형을 포함할 수 있다.

\index{기울기 (slope)}
\index{절편 (intercept)}
\index{측정 오차 (measurement error)}

만약 모수 {\tt inter}와 {\tt slope}이 잘못되면, 잔차는 더 커진다.
그래서 모수는 잔차를 최소화한다는 것이 직관적으로 의미가 있다.
\index{모수 (parameter)}

잔차 절대값, 잔차 제곱, 혹은 잔차 세제곱 최소화를 시도해볼만 하다;
하지만, 가장 흔한 선택은 제곱 잔차 합을 최소화하는 것이다. {\tt sum(res**2))}.

왜 그럴까요? 세가지 좋은 이유와 한가지 덜 중요한 이유가 있다.

\begin{itemize}

\item 제곱하게 되면 양수 잔차와 음수 잔차를 동일하게 처리하는 기능이 있는데, 보통 원하는 것이다.

\item 제곱은 큰 잔차에 더 많은 가중치를 주지만, 가장 큰 잔차가 항상 주도적인 경우에는 그렇게 많은 가중치를 주지는 않는다.

\item 만약 잔차가 상관관계가 없고 평균과 상수 (하지만 알려지지 않은 미지) 분산을 가진 정규분포라면,
최소제곱 적합은 또한 {\tt inter}와 {\tt slope}의 최대우도추정량이다. \url{https://en.wikipedia.org/wiki/Linear_regression}.  
\index{MLE}
  \index{최대우도추정량 (maximum likelihood estimator)}
\index{상관 (correlation)}

\item 제곱 잔차를 최소화하는 {\tt inter}와 {\tt slope} 값은 효과적으로 계산될 수 있다.

\end{itemize}

계산 효율성(computational efficiency)이 당면한 문제에 가장 적합한 방법을 선택하는 것보다 더 중요할 때 마지막 이유가 의미있다. 
이제는 더 이상 그럴지는 않다. 그래서, 제곱 잔차가 최소화하는 올바른 것인지만 고려한다.
\index{계산 방법 (computational methods)}
\index{제곱잔차 (squared residuals)}

예를 들어, {\tt xs}을 사용해서 {\tt ys} 값을 예측하려고 한다면,
과다 추정하는 것이 과소 추정하는 것보다 더 좋을 수도 (더 나쁠 수도) 있다.
이런 경우, 각 잔차에 대한 비용함수를 계산하고 전체 비용, {\tt sum(cost(res))}을 최소화한다.
하지만, 최소제곱 적합을 계산하는 것이 빠르고, 쉽고, 종종 충분히 만족스럽다.
\index{비용 함수 (cost function)}


\section{구현 (Implementation)}

{\tt thinkstats2}에 선형최소제곱을 시연하는 간단한 함수가 있다.
\index{LeastSquares}

\begin{verbatim}
def LeastSquares(xs, ys):
    meanx, varx = MeanVar(xs)
    meany = Mean(ys)

    slope = Cov(xs, ys, meanx, meany) / varx
    inter = meany - slope * meanx

    return inter, slope
\end{verbatim}

{\tt LeastSquares}는 시퀀스 {\tt xs}와 {\tt ys}을 인자로 받고 추정한 모수 {\tt inter}와
{\tt slope}을 반환한다. 동작방법에 관한 자세한 사항은 웹사이트 참조. \url{http://wikipedia.org/wiki/Numerical_methods_for_linear_least_squares}.
\index{모수 (parameter)}


{\tt thinkstats2}는 {\tt FitLine}를 제공하는데, {\tt inter} 와 {\tt slope}을 인자로 받아서 시퀀스 {\tt xs}에 대해서 적합선을 반환한다.
\index{FitLine}

\begin{verbatim}
def FitLine(xs, inter, slope):
    fit_xs = np.sort(xs)
    fit_ys = inter + slope * fit_xs
    return fit_xs, fit_ys
\end{verbatim}

이 함수를 사용해서 산모 연령 함수로 출생 체중에 대한 최소제곱을 계산할 수 있다.
\index{출생 체중 (birth weight)}
\index{체중 (weight)!출생 (birth)}
\index{연령 (age)}

\begin{verbatim}
    live, firsts, others = first.MakeFrames()
    live = live.dropna(subset=['agepreg', 'totalwgt_lb'])
    ages = live.agepreg
    weights = live.totalwgt_lb

    inter, slope = thinkstats2.LeastSquares(ages, weights)
    fit_xs, fit_ys = thinkstats2.FitLine(ages, inter, slope)
\end{verbatim}

추정한 절편과 기울기는 년마다 6.8 lbs, 0.017 lbs 이다.
이러 형태로 값을 해석하기는 어렵다: 절편은 산모 연령이 0 에서 신생아 기대 체중인데,
문맥상 의미가 없고, 기울기는 너무 작아서 쉽게 이해가 되지 않는다.
\index{기울기 (slope)}
\index{절편 (intercept)}
\index{dropna}
\index{NaN}


$x=0$에 절편을 제시하는 대신에, 절편을 평균에 제시하는 것이 종종 도움이 된다.
이 경우에, 평균 나이는 약 25세이고, 25세 산모에 대한 평균 아이 체중은 7.3 파운드다.
기울기는 년마다 0.27 온스(ounces) 즉, 10년마다 0.17 파운드가 된다.

\begin{figure}
% linear.py
%\centerline{\includegraphics[height=2.5in]{figs/linear1.pdf}}
\caption{Scatter plot of birth weight and mother's age with
a linear fit.}
\label{linear1}
\end{figure}

그림~\ref{linear1}에 적합선과 함께 출생 체중과 연령을 산점도로 보여준다.
이와 같이, 관계가 선형인지, 적합선이 관계를 나타내는 좋은 모형인지를 평가하는데, 그림을 그려보는 것은 좋은 생각이 된다. 

\index{출생 체중 (birth weight)}
\index{체중 (weight)!출생 (birth)}
\index{산점도 (scatter plot)}
\index{그림 (plot)!산점 (scatter)}
\index{모형 (model)}


\section{잔차 (Residuals)}
\label{residuals}

또다른 유용한 검정은 잔차를 플롯하여 그리는 것이다.
{\tt thinkstats2}에는 잔차를 계산하는 함수가 있다.
\index{잔차 (residuals)}

\begin{verbatim}
def Residuals(xs, ys, inter, slope):
    xs = np.asarray(xs)
    ys = np.asarray(ys)
    res = ys - (inter + slope * xs)
    return res
\end{verbatim}

{\tt Residuals}는 시퀀스 {\tt xs}와 {\tt ys}, 추정한 {\tt inter}와 {\tt slope}를 인자로 받는다. 실제값과 적합선 사이 차이를 반환한다.

\begin{figure}
% linear.py
%\centerline{\includegraphics[height=2.5in]{figs/linear2.pdf}}
\caption{Residuals of the linear fit.}
\label{linear2}
\end{figure}

잔차를 시각화하기 위해서, ~\ref{characterizing} 절에서 살펴본 것과 같이, 응답자를 연령으로 묶고, 각 집단에 백분위수를 계산한다. 
그림~\ref{linear2}에 각 연령 집단에 대한 25번째, 50번째, 75번째 백분위수가 있다.
중위수는 예상한 것처럼 거의 0이고, 사분위수 범위는 약 2 파운드다. 그래서, 만약 산모 연령을 알고 있다면, 1 파운드 내에서 대략 50\% 가능성으로 아이 체중을 추측할 수 있다.
\index{시각화 (visualization)}

이상적으로 이들 직선이 평평해서 잔차가 랜덤(random)임을 나타내고, 
평행해서 잔차 분산이 모든 연령 집단에 대해서 동일하다는 것을 나타내야 한다.
사실, 직선은 평행에 가깝워서 좋다; 하지만, 약간의 곡률(curvature)이 있어서 관계가 비선형임을 나타낸다.
그럼에도 불구하고, 선형 적합은 간단한 모형으로 특정 목적에 아마도 잘 부합한다.  

\index{모형 (model)}
\index{비선형 (nonlinear)}


\section{추정 (Estimation)}
\label{regest}

모수 {\tt slope}와 {\tt inter}는 표본에 기반한 추정값이다; 다른 추정값처럼, 표집 편의, 측정 오차, 표집 오차에 휘둘리기 쉽다. 
~\ref{estimation} 장에서 논의한 것처럼, 표집 편의는 비대표 표집(non-representative sampling)에 의해서, 측정 오차는 데이터 수집과 기록 오류에 의해서, 표집 오차는 전체 모집단보다 표본을 측정하는 결과로 발생한다.
\index{표집 편의 (sampling bias)}
\index{편의 (bias)!표집 (sampling)}
\index{측정 오차 (measurement error)}
\index{표집 오차 (sampling error)}
\index{추정 (estimation)}

표집 오차를 평가하기 위해서, ``만약 이 실험을 다시 실행한다면,
추정값에 얼마나 변동성이 예상될까?''
이러한 질문에 모의시험 실험을 진행하고, 추정값에 대한 표집 분포를 계산함으로써 대답할 수 있다.

\index{표집 오차 (sampling error)}
\index{표집 분포 (sampling distribution)}

데이터를 재표본추출(resampling)하으로써 실험을 모의시험할 수 있다; 즉, 관측된 임신을 마치 전체 모집단인 것처럼 처리해서 관측된 표본에서 복원 추출 방식으로 표본을 추출한다.
\index{모의 시험 (simulation)}
\index{복원 (replacement)}

\begin{verbatim}
def SamplingDistributions(live, iters=101):
    t = []
    for _ in range(iters):
        sample = thinkstats2.ResampleRows(live)
        ages = sample.agepreg
        weights = sample.totalwgt_lb
        estimates = thinkstats2.LeastSquares(ages, weights)
        t.append(estimates)

    inters, slopes = zip(*t)
    return inters, slopes
\end{verbatim}

{\tt SamplingDistributions} 함수는 인자로 정상 출산마다 한 줄(row)로 된 데이터프레임과 모의 시험 실험 횟수, {\tt iters}를 인자로 받는다. {\tt ResampleRows}을 사용해서 관측 임신을 재표본추출한다. 이미 {\tt SampleRows}를 살펴봤는데, 데이터프레임에서 무작위(random) 행을 선택한다. {\tt thinkstats2}는 또한 {\tt ResampleRows} 기능도 제공하는데, 원본과 동일한 크기 표본을 반환한다.
\index{데이터프레임 (DataFrame)}
\index{재표본추출 (resampling)}

\begin{verbatim}
def ResampleRows(df):
    return SampleRows(df, len(df), replace=True)
\end{verbatim}

재표본추출 후에, 모의 시험 표본을 사용해서 모수를 추정한다.
결과는 시퀀스 두개다: 추정 절편과, 추정 기울기.

\index{모수 (parameter)}

표준 오차와 신뢰구간을 출력해서 표집 분포를 요약한다.
\index{표집 분포 (sampling distribution)}

\begin{verbatim}
def Summarize(estimates, actual=None):
    mean = thinkstats2.Mean(estimates)
    stderr = thinkstats2.Std(estimates, mu=actual)
    cdf = thinkstats2.Cdf(estimates)
    ci = cdf.ConfidenceInterval(90)
    print('mean, SE, CI', mean, stderr, ci)
\end{verbatim}

{\tt Summarize}는 추정값과 실제값 시퀀을 인자로 받는다. 추정값 평균, 표준 오차, 그리고, 90\% 신뢰구간을 출력한다.
\index{표준 오차 (standard error)}
\index{신뢰 구간 (confidence interval)}

절편에 대해서, 평균 추정값은 6.83, 표준오차(SE)는 0.07, 90\% 신뢰구간(CI) (6.71, 6.94)이다. 좀더 간략한 형식으로, 추정 절편 0.0174, SE 0.0028, CI (0.0126, 0.0220)가 된다. 이 신뢰구간 하한과 상한 사이에 거의 두배 차이난다. 그래서, 개략적인 추정값으로 봐야한다.

%inter 6.83039697331 6.83174035366
%SE, CI 0.0699814482068 (6.7146843084406846, 6.9447797068631871)
%slope 0.0174538514718 0.0173840926936
%SE, CI 0.00276116142884 (0.012635074392201724, 0.021975282350381781)

추정값의 표집 오차를 시각화하기 위해서, 적합선을 모두 플롯으로 그리고, 연하게 채워서 각 연령에 대해서 90\% 신뢰구간을 플롯으로 그린다. 다음에 코드가 있다.

\begin{verbatim}
def PlotConfidenceIntervals(xs, inters, slopes,
                            percent=90, **options):
    fys_seq = []
    for inter, slope in zip(inters, slopes):
        fxs, fys = thinkstats2.FitLine(xs, inter, slope)
        fys_seq.append(fys)

    p = (100 - percent) / 2
    percents = p, 100 - p
    low, high = thinkstats2.PercentileRows(fys_seq, percents)
    thinkplot.FillBetween(fxs, low, high, **options)
\end{verbatim}

{\tt xs}는 산모 연령 시퀀스다. {\tt inters}와 {\tt slopes}는 {\tt SamplingDistributions}에서 생성된 추정 모수다. {\tt percent} 인자는 플롯을 얼마의 신뢰구간으로 그릴 것인지 나타낸다.

{\tt PlotConfidenceIntervals}는 {\tt inter}와 {\tt slope} 짝에 대해 적합선을 생성하고, 결과를 시퀀스 \verb"fys_seq"에 저장한다.
그리고 나서, {\tt PercentileRows}을 사용해서 {\tt x} 각 값에 대해서 {\tt y} 상한과 하한 백분위수를 선택한다. 
90\% 신뢰구간에 대해서, 5번째와 95번째 백분위수를 선택한다. 
{\tt FillBetween}은 두 직선 사이 공간을 채우는 다각형을 그린다.
\index{thinkplot}
\index{FillBetween}

\begin{figure}
% linear.py
%\centerline{\includegraphics[height=2.5in]{figs/linear3.pdf}}
\caption{50\% and 90\% confidence intervals showing variability in the
  fitted line due to sampling error of {\tt inter} and {\tt slope}.}
\label{linear3}
\end{figure}

그림~\ref{linear3}에는 산모 연령에 대한 함수로 출생 체중에 적합된 곡선에 대한 50\% 와 90\% 신뢰구간이 있다. 구역의 수직폭(vertical width)이 표집 오차에 대한 효과를 표현한다; 효과가 평균 주변 값에 대해 더 작고, 극단값에 대해 더 크다.

\section{적합도 (Goodness of fit)}
\label{goodness}
\index{적합도 (goodness of fit)}

선형 모형 품질, 즉 {\bf 적합도 (goodness of fit)}를 측정하는 방법이 몇가지 있다. 가장 간단한 방법은 잔차의 표준편차다.

\index{표준 편차 (standard deviation)}
\index{모형 (model)}

만약 예측을 위해 선형 모형을 사용한다면, {\tt Std(res)}는 예측의 제곱근 평균제곱오차(root mean squared error, RMSE)다.
예를 들어, 만약 출생 체중을 추측하는데 산모 연령을 사용한다면, 추측 RMSE는 1.40 lbs가 된다.
\index{출생 체중 (birth weight)}
\index{체중 (weight)!출생 (birth)}

만약 산모 나이를 모른 상태에서 출생 체중을 추측한다면, 
추측 RMSE는 {\tt Std(ys)}로, 1.41 lbs다.  
그래서, 이 예제에서 산모 연령을 알고 있는 것은 그다지 예측력을 향상시키지 못한다.
\index{예측 (prediction)}

적합도를 측정하는 또 다른 방식은 $R^2$로 표기하고, ``R-제곱(R-squared)''라고 부르는 {\bf 결정계수 (coefficient of determination)}다.

\index{결정계수 (coefficient of determination)}
\index{r-제곱 (r-squared)}

\begin{verbatim}
def CoefDetermination(ys, res):
    return 1 - Var(res) / Var(ys)
\end{verbatim}

{\tt Var(res)}는 모형을 사용한 추측 MSE 이고, {\tt Var(ys)}는 
모형없는 MSE 다. 그래서, 비율이 만약 모형을 사용한다면 남게되는 MSE 부분이다. $R^2$ 는 모형이 제거하는 MSE 부분이 된다.
\index{MSE}

출산 체중과 산모 나이에 대해, $R^2$ 값은 0.0047 으로, 산모 나이가 출생 체중에 있는 분산 약 0.5\%만 예측한다는 의미가 된다.

결정계수와 피어슨 상관계수 사이에 간단한 관계가 있다: $R^2 = \rho^2$.
예를 들어, 만약 $\rho$ 가 0.8 혹은 -0.8 이라면, $R^2 = 0.64$가 된다.
\index{피어슨 상관계수 (Pearson coefficient of correlation)}

$\rho$와 $R^2$가 종종 관계 강도를 정량화하는데 사용될지라도,
예측력(predictive power)에 대해서 해석하기는 쉽지 않다.
저자 견해로는, {\tt Std(res)}가 예측 품질을 가장 잘 표현한다고 본다. 특히, 만약 {\tt Std(ys)}와 연관되어 표현되면 그렇다. 

\index{결정계수 (coefficient of determination)}
\index{r-제곱 (r-squared)}

예를 들어, SAT(미국 대학 입학시험에 사용되는 표준국가시험) 타당성에 대해서 얘기할 때, 종종 사람들은 SAT 점수와 다른 지능(IQ) 측정값 사이에 상관을 얘기한다.
\index{SAT}
\index{지능 (IQ)}

한 연구에 의하면, SAT 점수와 IQ 점수 사이에 피어슨 상관은 $\rho=0.72$ 으로, 강한 상관처럼 보인다.
하지만, $R^2 = \rho^2 = 0.52$ 이여서 SAT 점수는 IQ 분산의 단지 52\%만 설명한다.

IQ 점수는 {\tt Std(ys) = 15} 로 정규화된다. 그래서, 

\begin{verbatim}
>>> var_ys = 15**2
>>> rho = 0.72
>>> r2 = rho**2
>>> var_res = (1 - r2) * var_ys
>>> std_res = math.sqrt(var_res)
10.4096
\end{verbatim}

IQ 예측하는데 SAT 점수를 사용하는 것은 RMSE를 15 점에서 10.4 점으로 줄인다. 상관계수 0.72 는 RMSE 를 줄이는데 단지 31\% 기여한다.

만약 상관계수가 감명적으로 보인다면, $R^2$이 MSE 축소에 더 나은 지표가 되고, RMSE 축소가 예측력에 대한 더 나은 지표다.

\index{결정계수 (coefficient of determination)}
\index{r-제곱 (r-squared)}
\index{예측 (prediction)}


\section{선형 모형 검정 (Testing a linear model)}

출생 체중에 대한 산모 연령 효과가 작고, 거의 예측력이 없다.
그래서, 외관 관계(apparent relationship)가 유연에 의해서 가능한 것인가?
선형적합 결과를 검정하는 방법이 몇개 있다.
\index{출생 체중 (birth weight)}
\index{체중 (weight0!출생 (birth)}
\index{모형 (model)}
\index{선형 모형 (linear model)}

한 선택옵션은 MSE에 외관 축소(apparent reduction)가 우연에 의한 것인지 검정하는 것이다. 이 경우에 검정 통계량은 $R^2$ 이고, 귀무 가설은 변수 간 관계가 없다가 된다. ~\ref{corrtest} 절에서 산모 연령과 출생 체중 간 상관을 검정했을 처럼, 귀무가설을 순열(permutation)을 통해서 모의 시험할 수 있다. 사실, $R^2 = \rho^2$, 이기 때문에, $R^2$ 단측 검정은 $\rho$ 양측 검정과 상응한다. 이미 이 검정을 수행했고, $p < 0.001$ 라는 것을 발견했다. 그래서, 산모 연령과 출생 체중 간 외관 관계는 통계적 유의성이 있다고 결론낸다.

\index{귀무 가설 (null hypothesis)}
\index{순열 (permutation)}
\index{결정계수 (coefficient of determination)}
\index{r-제곱 (r-squared)}
\index{유의성 (significant)} 
\index{통계적 유의성 (statistically significant)}

또 다른 접근법은 외관 기울기(apparent slope)가 우연에 의한 것인지 검정하는 것이다. 귀무가설은 기울기가 실제로 0 이다는 것이다; 이 경우에, 출생 체중을 평균 근처에 확률변동(random variation)으로 모형화할 수 있다. 다음에 이 모형에 대한 HypothesisTest 가 있다.
\index{HypothesisTest}
\index{모형 (model)}

\begin{verbatim}
class SlopeTest(thinkstats2.HypothesisTest):

    def TestStatistic(self, data):
        ages, weights = data
        _, slope = thinkstats2.LeastSquares(ages, weights)
        return slope

    def MakeModel(self):
        _, weights = self.data
        self.ybar = weights.mean()
        self.res = weights - self.ybar

    def RunModel(self):
        ages, _ = self.data
        weights = self.ybar + np.random.permutation(self.res)
        return ages, weights
\end{verbatim}

데이터는 연령과 체중 시퀀스로 표현된다.
검정 통계량은 {\tt LeastSquares}로 추정된 기울기다.
귀무 가설 모형은 모든 아기들의 평균 체중과 평균에 대한 편차로 표현된다.
모의 시험 데이터를 생성하기 위해서, 편차를 순열로 배치하고, 평균에 더한다.

\index{편차 (deviation)}
\index{귀무가설 (null hypothesis)}
\index{순열 (permutation)}

다음에 가설 검정을 수행하는 코드가 있다.

\begin{verbatim}
    live, firsts, others = first.MakeFrames()
    live = live.dropna(subset=['agepreg', 'totalwgt_lb'])
    ht = SlopeTest((live.agepreg, live.totalwgt_lb))
    pvalue = ht.PValue()
\end{verbatim}

p-값이 $0.001$ 보다 작다, 그래서 설사 추정된 기울기가 작지만, 우연에 의한 것 같지는 않다.

\index{p-값 (p-value)}
\index{dropna}
\index{NaN}

귀무가설을 모의 시험함으로써 p-값을 추정하는 것이 엄격하게는 맞다. 하지만, 더 간단한 대안이 있다. 이미 ~\ref{regest} 절에서 기울기 표집 분포를 계산한 것을 기억하라. 이를 위해서, 관측된 기울기가 맞다고 가정하고 재표본추출(resampling)으로 실험을 모의 시험했다.

\index{귀무가설 (null hypothesis)}

그림~\ref{linear4}에는 ~\ref{regest}절과 귀무가설 아래에서 생성된 기울기 분포에서 나온 기울기 표집 분포가 있다.
표집 분포가 추정된 기울기 약 0.017 lbs/년을 중심으로 있고, 귀무가설 아래 기울기가 약 0 을 중심으로 있다; 하지만, 그것을 제외하고 분포는 동일한다. 분포는 또한 ~\ref{CLT}절에서 살펴보게될 이유로 대칭이다. 
 
\index{대칭 (symmetric)}
\index{표집 분포 (sampling distribution)}

\begin{figure}
% linear.py
%\centerline{\includegraphics[height=2.5in]{figs/linear4.pdf}}
\caption{The sampling distribution of the estimated
slope and the distribution of slopes
generated under the null hypothesis.  The vertical lines are at 0
and the observed slope, 0.017 lbs/year.}
\label{linear4}
\end{figure}

그래서, p-값을 두 방식으로 추정할 수 있다:
\index{p-값 (p-value)}

\begin{itemize}

\item 귀무가설 아래서 기울기가 관측 기울기를 초과할 확률을 계산한다.
\index{귀무가설 (null hypothesis)}

\item 표집분포에서 기울기가 0 이하인 확률을 계산한다. (만약 추정 기울기가 음수라면, 표집 분포에서 기울기가 0을 초과하는 확률을 계산할 것이다.)

\end{itemize}

두번째 선택옵션은 더 쉬운데 이유는 어떻든 정상적으로 모수의 표집 분포를 계산하고자 하기 때문이다. 그리고 표본 크기가 작지 않고, {\em 그리고} 잔차 분포가 기울지 않았다면 좋은 근사(approximation)가 된다. 그때조차도, p-값이 정교할 필요가 없기 때문에, 대체로 만족스럽다.

\index{왜도 (skewness)}
\index{모수 (parameter)}

다음에 표집분포를 사용해서 기울기 p-값을 추정하는 코드가 있다.
\index{표집 분포 (sampling distribution)}

\begin{verbatim}
    inters, slopes = SamplingDistributions(live, iters=1001)
    slope_cdf = thinkstats2.Cdf(slopes)
    pvalue = slope_cdf[0]
\end{verbatim}

다시한번, $p < 0.001$이 나온다.  


\section{가중 재표본추출 (Weighted resampling)}
\label{weighted}

지금까지 NSFG 데이터를 마치 대표 표본인 것처럼 다루었다. 하지만, ~\ref{nsfg} 절에서 언급한 것 같이, 대표 표본은 아니다. 의도적으로 조사는 몇몇 집단을 오버샘플링(oversampling) 하는데 이유는 통계적으로 유의적인 결과를 도출할 확률을 높이기 위해서다; 즉, 이들 집단에 대해서 검정력을 향상하기 위해서다.
\index{유의성 (significant)} 
\index{통계적 유의성 (statistically significant)}

조사설계가 많은 목적에 대해서 유용하지만, 표집 과정을 고려하지 않고, 일반 모집단에 대한 값을 추정하는데 표본을 사용할 수 있다는 것을 의미하지는 않는다.

각 응답자에 대해서, NSFG 데이터는 {\tt finalwgt} 변수가 있는데, 응답자가 대표하는 일반 모집단에 속한 사람 숫자다. 이 값은 {\bf 표집 가중치 (sampling weight}, 즉 ``가중치 (weight)''라고 불린다.
\index{표집 가중치 (sampling weight)}
\index{가중치 (weight)}
\index{가중 재표본추출 (weighted resampling)}
\index{재표본추출 (resampling)!가중 (weighted)}

예제로, 만약 3억 인구를 가진 나라에서 100,000 명을 조사한다면, 각 응답자는 3,000 명을 대표한다. 만약 한 집단을 2배 오버샘플링한다면, 오버샘플링된 집단에 각 사람은 더 적은 가중치(약 1500)가 된다.

오버샘플링을 보정하기 위해서, 재표본추출(resampling)을 사용할 수 있다; 즉, 표집 가중치에 비례하는 확률을 사용해서 조사자료에서 표본을 추출한다.
그리고 나서, 추정하려고 하는 임의 정량정보에 대해서 표집 분포, 표본 오차, 신뢰구간을 생성할 수 있다. 예제로, 평균 출생 체중을 표본 가중치를 두고, 안두고 추정할 것이다.

\index{표본 오차 (standard error)}
\index{신뢰 구간 (confidence interval)}
\index{출생 체중 (birth weight)}
\index{체중 (weight)!출생 (birth)}
\index{표집 분포 (sampling distribution)}
\index{오버샘플링 (oversampling)}

~\ref{regest}절에서, {\tt ResampleRows}를 살펴봤다. 동일한 확률을 각 행에 두고 데이터프레임에서 행을 추출한다. 이제 동일한 것을 표집 가중치에 비례한 확률을 사용해서 수행할 필요가 있다. {\tt ResampleRowsWeighted}는 데이터프레임을 인자로 받고, {\tt finalwgt}에 있는 가중치에 따라 행을 재표본 추출하고, 재표본추출된 행을 포함하는 데이터프레임을 반환한다.

\index{데이터프레임 (DataFrame)}
\index{재표본추출 (resampling)}

\begin{verbatim}
def ResampleRowsWeighted(df, column='finalwgt'):
    weights = df[column]
    cdf = Cdf(dict(weights))
    indices = cdf.Sample(len(weights))
    sample = df.loc[indices]
    return sample
\end{verbatim}

{\tt weights}는 시리즈다; 시리즈를 딕셔너리로 전환하는 것은 인덱스를 가중치로 매핑한다. {\tt cdf}에서 값은 인덱스고, 확률은 가중치에 비례한다.

{\tt indices}는 행인덱스 시퀀스다; {\tt sample}은 선택된 행을 담고 있는 데이터프레임이다. 복원 표본 추출을 했기 때문에, 동일한 행이 한번이상 나올 수 있다.
\index{Cdf}
\index{복원 (replacement)}

이제, 가중치를 두고, 두지 않고 재표본추출 효과를 비교할 수 있다.
가중치를 두지 않고, 다음과 같이 표집분포를 생성한다.
\index{표집 분포 (sampling distribution)}

\begin{verbatim}
    estimates = [ResampleRows(live).totalwgt_lb.mean()
                 for _ in range(iters)]
\end{verbatim}

가중치를 두면, 다음과 같다.

\begin{verbatim}
    estimates = [ResampleRowsWeighted(live).totalwgt_lb.mean()
                 for _ in range(iters)]
\end{verbatim}

다음 표에 요약결과가 나와있다.

\begin{center}
\begin{tabular}{|l|c|c|c|}
\hline
                    &  mean birth   & standard  &  90\% CI  \\ 
                    &  weight (lbs) & error     &           \\ 
\hline
Unweighted          &  7.27  &  0.014  &  (7.24, 7.29)  \\ 
Weighted            &  7.35  &  0.014  &  (7.32, 7.37)  \\ 
\hline
\end{tabular}
\end{center}

%mean 7.26580789518
%stderr 0.0141683527792
%ci (7.2428565501217079, 7.2890814917127074)
%mean 7.34778034718
%stderr 0.0142738972319
%ci (7.3232804012858885, 7.3704916897506925)

예제애서, 가중치를 둔 효과는 적지만 무시할만하지는 않다.
가중치를 두고, 두지 않고 추정된 평균에 차이는 약 0.08 파운드, 1.3 온스다.
차이가 추정값 표준 오차(0.014 파운드)보다 실질적으로 더 크다. 함축하는 바는 차이는 우연적 요소에 의한 것은 아니라는 것이다.
\index{표준 오차 (standard error)}
\index{신뢰 구간 (confidence interval)}


\section{연습문제}

A solution to this exercise is in \verb"chap10soln.ipynb"

\begin{exercise}

Using the data from the BRFSS, compute the linear least squares
fit for log(weight) versus height.
How would you best present the estimated parameters for a model
like this where one of the variables is log-transformed?
If you were trying to guess
someone's weight, how much would it help to know their height?
\index{Behavioral Risk Factor Surveillance System}
\index{BRFSS}
\index{model}

Like the NSFG, the BRFSS oversamples some groups and provides
a sampling weight for each respondent.  In the BRFSS data, the variable
name for these weights is {\tt totalwt}.
Use resampling, with and without weights, to estimate the mean height
of respondents in the BRFSS, the standard error of the mean, and a
90\% confidence interval.  How much does correct weighting affect the
estimates?
\index{confidence interval}
\index{standard error}
\index{oversampling}
\index{sampling weight}
\end{exercise}


\section{용어 사전}

\begin{itemize}

\item 선형적합 (linear fit): 변수 사이 관계를 모형화하려는 직선.
\index{선형적합 (linear fit)}

\item 최소제곱적합 (least squares fit): 잔차제곱합을 최소화하는 데이터셋 모형.
\index{최소제곱적합 (least squares fit)}

\item 잔차 (residual): 실제값과 모형값 편차.
\index{잔차 (residuals)}

\item 적합도 (goodness of fit): 모형이 데이터에 얼마나 잘 적합하는지에 대한 척도.
\index{적합도 (goodness of fit)}

\item 결정계수 (coefficient of determination): 적합도를 계량화하려는 통계량.
\index{결정계수 (coefficient of determination)}

\item 표집 가중치 (sampling weight): 
표본이 모집단 어느 부분을 대표하는지 나타내는데 표본 관측과 연관된 값.
\index{표집 가중치 (sampling weight)}

\end{itemize}

